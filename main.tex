\documentclass[twocolumn,astrosymb,twocolappendix]{aastex631}

%% The default is a single spaced, 10 point font, single spaced article.
%% There are 5 other style options available via an optional argument. They
%% can be invoked like this:
%%
%% \documentclass[arguments]{aastex631}
%% 
%% where the layout options are:
%%
%%  twocolumn   : two text columns, 10 point font, single spaced article.
%%                This is the most compact and represent the final published
%%                derived PDF copy of the accepted manuscript from the publisher
%%  manuscript  : one text column, 12 point font, double spaced article.
%%  preprint    : one text column, 12 point font, single spaced article.  
%%  preprint2   : two text columns, 12 point font, single spaced article.
%%  modern      : a stylish, single text column, 12 point font, article with
%% 		  wider left and right margins. This uses the Daniel
%% 		  Foreman-Mackey and David Hogg design.
%%  RNAAS       : Supresses an abstract. Originally for RNAAS manuscripts 
%%                but now that abstracts are required this is obsolete for
%%                AAS Journals. Authors might need it for other reasons. DO NOT
%%                use \begin{abstract} and \end{abstract} with this style.
%%
%% Note that you can submit to the AAS Journals in any of these 6 styles.
%%
%% There are other optional arguments one can invoke to allow other stylistic
%% actions. The available options are:
%%
%%   astrosymb    : Loads Astrosymb font and define \astrocommands. 
%%   tighten      : Makes baselineskip slightly smaller, only works with 
%%                  the twocolumn substyle.
%%   times        : uses times font instead of the default
%%   linenumbers  : turn on lineno package.
%%   trackchanges : required to see the revision mark up and print its output
%%   longauthor   : Do not use the more compressed footnote style (default) for 
%%                  the author/collaboration/affiliations. Instead print all
%%                  affiliation information after each name. Creates a much 
%%                  longer author list but may be desirable for short 
%%                  author papers.
%% twocolappendix : make 2 column appendix.
%%   anonymous    : Do not show the authors, affiliations and acknowledgments 
%%                  for dual anonymous review.
%%
%% these can be used in any combination, e.g.
%%
%% \documentclass[twocolumn,linenumbers,trackchanges]{aastex631}
%%
%% AASTeX v6.* now includes \hyperref support. While we have built in specific
%% defaults into the classfile you can manually override them with the
%% \hypersetup command. For example,
%%
%% \hypersetup{linkcolor=red,citecolor=green,filecolor=cyan,urlcolor=magenta}
%%
%% will change the color of the internal links to red, the links to the
%% bibliography to green, the file links to cyan, and the external links to
%% magenta. Additional information on \hyperref options can be found here:
%% https://www.tug.org/applications/hyperref/manual.html#x1-40003
%%
%% Note that in v6.3 "bookmarks" has been changed to "true" in hyperref
%% to improve the accessibility of the compiled pdf file.
%%
%% If you want to create your own macros, you can do so
%% using \newcommand. Your macros should appear before
%% the \begin{document} command.
%%
\newcommand{\vdag}{(v)^\dagger}
\newcommand\aastex{AAS\TeX}
\newcommand\latex{La\TeX}
% \newcommand{\sbunit}{mag~arcsec$^{-2}$}
\newcommand{\sbunit}{\mathrm{mag\ arcsec}^{-2}}
\newcommand{\sbeff}{\overline{\mu}_{\mathrm{eff}}(g)}
\newcommand{\jiaxuan}[1]{\textcolor{orange}{\textbf{Jiaxuan: #1}}}


\newcommand{\code}[1]{\textbf{\texttt{#1}}}
\newcommand{\sersic}{S\'ersic}
\usepackage{CJKutf8}
\usepackage{bm}
\usepackage{appendix}
\usepackage{amsmath,amssymb}

%% Reintroduced the \received and \accepted commands from AASTeX v5.2
\received{\today}
\revised{\today}
\accepted{\today}

\submitjournal{ApJ}

%% alias for citations
\defcitealias{Greco2018}{G18}

% \def\G18{\citetalias{Greco2018}}

\shorttitle{UDGs in MW analogs}
\shortauthors{Li et al.}
%%
%% You can add a light gray and diagonal water-mark to the first page 
%% with this command:
%% \watermark{text}
%% where "text", e.g. DRAFT, is the text to appear.  If the text is 
%% long you can control the water-mark size with:
%% \setwatermarkfontsize{dimension}
%% where dimension is any recognized LaTeX dimension, e.g. pt, in, etc.
%%
%%%%%%%%%%%%%%%%%%%%%%%%%%%%%%%%%%%%%%%%%%%%%%%%%%%%%%%%%%%%%%%%%%%%%%%%%%%%%%%%
\graphicspath{{./}{figures/}}
%% This is the end of the preamble.  Indicate the beginning of the
%% manuscript itself with \begin{document}.

\begin{document}
\begin{CJK*}{UTF8}{gbsn}

\title{Ultra Diffuse Galaxies associated with Milky-Way Analogs}

% \correspondingauthor{Jiaxuan Li}
\author[0000-0001-9592-4190]{Jiaxuan Li (李嘉轩)}
\affiliation{Department of Astrophysical Sciences, 4 Ivy Lane, Princeton University, Princeton, NJ 08544, USA}

\author[0000-0002-5612-3427]{Jenny E. Greene}
\affiliation{Department of Astrophysical Sciences, 4 Ivy Lane, Princeton University, Princeton, NJ 08544, USA}

\author[0000-0003-4970-2874]{Johnny Greco}
\affiliation{Department of Astrophysical Sciences, 4 Ivy Lane, Princeton University, Princeton, NJ 08544, USA}
\affiliation{Center for Cosmology and AstroParticle Physics (CCAPP), The Ohio State University, Columbus, OH 43210, USA}

\author[0000-0003-1385-7591]{Song Huang (黄崧)}
\affiliation{Department of Astrophysical Sciences, 4 Ivy Lane, Princeton University, Princeton, NJ 08544, USA}
\affiliation{Department of Astronomy and Tsinghua Center for Astrophysics, Tsinghua University, Beijing 100084, China}

\author[0000-0002-1841-2252]{Rachael Beaton}
\affiliation{Department of Astrophysical Sciences, 4 Ivy Lane, Princeton University, Princeton, NJ 08544, USA}
\author[0000-0002-2991-9251]{Kirsten Casey}
\affiliation{Center for Cosmology and AstroParticle Physics (CCAPP), The Ohio State University, Columbus, OH 43210, USA}
\author[0000-0002-1841-2252]{Shany Danieli}
\affiliation{Department of Astrophysical Sciences, 4 Ivy Lane, Princeton University, Princeton, NJ 08544, USA}
\author[0000-0002-1841-2252]{Andy Goulding}
\affiliation{Department of Astrophysical Sciences, 4 Ivy Lane, Princeton University, Princeton, NJ 08544, USA}
\author[0000-0002-2704-5028]{Remy Joseph}
\affiliation{Department of Astrophysical Sciences, 4 Ivy Lane, Princeton University, Princeton, NJ 08544, USA}
\author[0000-0002-1841-2252]{Erin Kado-Fong}
\affiliation{Department of Astrophysical Sciences, 4 Ivy Lane, Princeton University, Princeton, NJ 08544, USA}
\author[0000-0002-8873-5065]{Peter Melchior}
\affiliation{Department of Astrophysical Sciences, 4 Ivy Lane, Princeton University, Princeton, NJ 08544, USA}

%% Note that the \and command from previous versions of AASTeX is now
%% depreciated in this version as it is no longer necessary. AASTeX 
%% automatically takes care of all commas and "and"s between authors names.

%% AASTeX 6.31 has the new \collaboration and \nocollaboration commands to
%% provide the collaboration status of a group of authors. These commands 
%% can be used either before or after the list of corresponding authors. The
%% argument for \collaboration is the collaboration identifier. Authors are
%% encouraged to surround collaboration identifiers with ()s. The 
%% \nocollaboration command takes no argument and exists to indicate that
%% the nearby authors are not part of surrounding collaborations.

%% Mark off the abstract in the ``abstract'' environment. 
\begin{abstract}

\end{abstract}

\keywords{Low surface brightness galaxies (940), Dwarf galaxies (416), Galaxy properties (615), Galaxy abundances (574)}


\section{Introduction} \label{sec:intro}

Since the discovery of ultra-diffuse galaxies (UDGs, defined to have $r_e > 1.5$ kpc and $\mu(g,0)>24.0\ \mathrm{mag\ arcsec^{-2}}$) in the Coma cluster \citep{vanDokkum2015}, there have been a variety of studies for these extreme systems ranging from their dark matter content (van Dokkum et al. 2018) to their globular cluster populations (Danieli et al. 2022). A central question related to UDGs is what physical mechanisms are responsible for their formation. Before we can answer this question, it is important to reliably determine in what environment they live.
Past UDG studies focused on rich galaxy groups and clusters (e.g., Yagi et al. 2016, Zatrisky et al. 2019), which provides distance information. Although UDGs are also found in field environments (Prole et al. 2019, Tanoglidis et al. 2021, Kadowaki et al. 2021), most studies of field UDGs lack distances and thus are statistical in nature. UDGs in groups and clusters are red and quenched; their blue counterparts are typically found in field environments (for some exceptions see Rom\'{a}n et al. 2019, Prole et al. 2019). This suggests that environment plays a crucial role in the formation and quenching of UDGs. Tidal effects and ram-pressure stripping explains the high quenched fraction in clusters and groups, but it is unclear whether field UDGs quench because of internal processes (feedback) and/or external processes (as splashback satellites, Benavides et al. 2021). It is also unclear whether UDGs belong to a special class of galaxy or are just a natural extension of dwarf systems to the lower surface brightness end. A larger UDG sample, especially in low-density environments, is needed to test these hypotheses.

The best knowledge we have about dwarf galaxies at the very faint end was derived from systems in the Local Volume (e.g., Simon 2019, Carlsten et al. 2021). To better compare UDGs with other dwarf systems, we turned to looking for UDGs in the vicinity of Milky Way analogs at $0.01 < z < 0.04$. Until now, there are only $\sim 10$ confirmed UDGs associated with MW analogs (Cohen et al. 2018, Mao et al. 2021). The occurrence rate of UDGs in these low-density environments is unknown, and whether UDGs are quenched in such a regime has not been explored. With a sample of UDGs around MW-like hosts, we can compare the structural parameters, quenched fractions, nucleation fractions, and radial distributions with normal satellites in the nearby universe \citep[e.g.,][]{SAGA-II,CarlstenELVES2022}, to shed light on questions about the formation and evolution of the UDGs.


Continuing the work by \citep{Greco2018}, we conducted a systematic search for low surface brightness galaxies (LSBGs) in Hyper Suprime-Cam survey (HSC) imaging data (Li et al., in prep). In order to study the formation and evolution of UDGs in low-density environments, we first select host galaxies at $0.01 < z < 0.04$ with Milky-Way masses ($10.2 < \log M_*/M_\odot < 11.2$) in the NASA-Sloan Atlas. We matched our HSC LSBGs catalog with the MW-like hosts catalog. We assign an LSBG to a host if their angular separation is smaller than the projected virial radius of the host. Then we select a UDG sample by requiring the $g$-band average surface brightness within the effective radius\footnote{This surface brightness threshold is equivalent to $\mu_0(g) > 24.0\ \mathrm{mag\ arcsec^{-2}}$ (van Dokkum et al. 2015) for S\'{e}rsic model with $n=1$.} $\overline{\mu_e} > 24.4\ \mathrm{mag\ arcsec^{-2}}$ and effective radius $R_e > 1.5$ kpc, under the assumption that all matched LSBGs are physically associated with the hosts. Following these steps, we have identified $\sim$ 250 UDG candidates to be associated with Milky Way-like hosts in projection. They span a wide range of surface brightness, size, color, nucleation, and host properties, making this sample ideal for studying UDG spatial distributions, quenched fractions, mass-size relations, nucleation fractions, and their dependence on host properties.


MW satellites: \citep{Nashimoto2022}

In this work, we use the circularized effective radius $r_{\rm eff}$, defined as the $r_{\rm eff} = r_{\rm eff, sma} \sqrt{b/a}$, where $r_{\rm eff, sma}$ is the effective radius along the semi-major axis of the aligned elliptical isophotes, and $b/a$ is the axis ratio of the isophotes.

We adopt a $\Lambda$CDM cosmology from \citet{Planck15} with $\Omega_{\rm m}= 0.307$ and $H_0 = 67.7\ $km s$^{-1}$ Mpc$^{-1}$. We use the AB system \citep{Oke1983} for magnitudes. The stellar mass used in this work is based on a \citet{Chabrier2003} initial mass function.

\section{Data} \label{sec:data}
\subsection{Hyper Suprime-Camera data}
The Hyper Suprime-Camera Subaru Strategic Program Survey (\citealt{Aihara2018}; hereafter HSC survey)\footnote{\url{https://hsc-release.mtk.nao.ac.jp/doc/}} is an optical imaging survey using the 8.2-m Subaru telescope and the Hyper Suprime-Camera \citep{Miyazaki2012, Miyazaki2018}. The \texttt{Wide} layer is designed to cover $\sim 1000\ \rm{deg}^{2}$ of the sky in five broad bands ($grizy$), reaching a depth of $g=26.6$ mag, $r=26.2$ mag and $i=26.2$ mag ($5\sigma$ point source). HSC data are processed using \code{hscPipe}\footnote{\url{https://hsc.mtk.nao.ac.jp/pipedoc_e/}} \citep{Bosch2018}, which is a customized version of the Large Synoptic Survey Telescope (LSST) pipeline \citep{LSST-pipeline}\footnote{\url{https://pipelines.lsst.io/}}. 

In this work, we use the \code{Wide} layer data from Public Data Release 2 (PDR2, also known as \code{S18A}) of HSC \citealt{Aihara2018}. It covers $\sim 300\ \rm{deg}^2$ in all five bands, which is 1.5 times larger than the dataset analyzed in \citetalias{Greco2018}. One of the key improvements made in \code{S18A} is the sky background subtraction. Compared with previous data releases, \code{S18A} adopted a full focal plane sky subtraction algorithm to overcome the over-subtraction of local sky background around bright objects \citep{Aihara2018,Li2021}. The unprecedented depth and careful sky subtraction makes \code{S18A} an ideal dataset to search for low surface brightness galaxies. 

%HSC \code{S18A} also provides bitmasks indicating bad pixels, cosmic rays, edges of CCDs and pixels with source detection, helping us generate image masks when extracting surface brightness profiles (Section \ref{sec:hsc_methods}). In this paper, we use the \code{WIDE} layer data from \code{S18A} (\code{PDR2}). It covers $\sim 300$ deg$^2$ in all five bands. 

\subsection{NASA-Sloan Atlas}
We use the NASA-Sloan Atlas (NSA \footnote{\url{http://nsatlas.org}}, \citealt{Blanton2005,Blanton2011}) to select galaxies analogous to the Milky Way. NSA is a catalog of parameters of local galaxies derived from the Sloan Digital Sky Survey \citep[SDSS,][]{York2000}. We use the new version of NSA catalog (\code{v1\_0\_1}\footnote{\url{https://www.sdss.org/dr13/manga/manga-target-selection/nsa/}}) which contains about $640,000$ galaxies out to $z < 0.15$. It also includes elliptical Petrosian aperture photometry for galaxies, which is considered to be more reliable than the photometry used in older versions. In this paper, we use the stellar mass derived from the ellpitical Petrosian photometry using \code{kcorrect v4\_2}. The redshifts of galaxies in NSA come from several spectroscopic surveys, gas surveys, or direct distance measurements. 



\section{LSBG search in HSC S18A}
Given an overview of the steps here.

\subsection{Source Detection}\label{sec:detection}
\citetalias{Greco2018} performed a search for extended LSBGs in the first $\sim 200$ deg$^2$ of HSC survey. We continued the work in \citetalias{Greco2018} and extended the search to HSC \code{S18A} data which covers $\sim 300\ \rm{deg}^{2}$ and has much better sky subtraction compared to \code{S16A}. We follow the same philosophy of source detection as in \citetalias{Greco2018}, but make several updates to accommodate \code{S18A} data. Below we summarize the main steps of the search and emphasize the updates made to improve the overall completeness and purity. We refer interested readers to \citetalias{Greco2018} for more details. Our source detection pipeline \code{hugs}\footnote{\url{https://github.com/johnnygreco/hugs}} is also publically available.

\begin{enumerate}
    \item \textbf{Bright source removal}. Bright sources and their associated LSB lights can mimic objects of interest and obstruct effective detection of LSBGs. In this step, we replace pixels related to bright sources with sky noise. The bright sources and their diffuse lights are detected by applying a high thresholding and a low thresholding to the image. A diffuse light component is associated with a bright source if more than 15\% of its pixels are above the high threshold. In this way, we generate a footprint of bright sources in the image.
    
    Since the sky subtraction in \code{S18A} is significantly better than in previous data releases, LSB features are well conserved after subtracting the sky. Therefore, it makes more sense to set the thresholding based on surface brightness, instead of a certain sigma value above sky background (as in \citetalias{Greco2018}). We set the high threshold to $\mu_{\rm high} = 22\ \mathrm{mag\ arcsec^{-2}}$ to capture all bright sources above this surface brightness, and the low threshold to $\mu_{\rm low} = 24.5\ \mathrm{mag\ arcsec^{-2}}$ to capture associated diffuse lights. 
    
    %Unlike in \citetalias{Greco2018}, we do not smooth the image prior to thresholding. 
    After this step, there are still a number of small low surface brightness objects, which are typically marginally resolved galaxies or just pixels with high noise fluctuations. We detect and remove them since our main goal is to detect extended LSBGs. We run \code{sep}\footnote{\url{https://sep.readthedocs.io/en/v1.1.x/}} \citep{Barbary2016} on the image cleaned as above. Based on the segmantation map, we generate a mask for sources smaller than $r_{\rm min} = 2\arcsec$. This step reduces the number of blended sources. These values ($\mu_{\rm high},\ \mu_{\rm low},\ r_{\rm min})$ are chosen by trial and error. We also get feedbacks from the completeness tests. Not smoothing the images gives us marginal improvement on completeness. 

    In the end, we replace pixels within the bright source footprint and small source mask with sky noise. The detection is done in $i$-band, and we use the same mask for all $gri$-bands. This step effectively cleans out objects and features that hinder the detection of LSBGs. 
    
    \item \textbf{Source Extraction}. We use \code{Source Extractor} \citep{Bertin1996} to detect sources on the ``cleaned'' images produced in the previous step. This step remains largely the same as in \citetalias{Greco2018}. The images are convoled with a Gaussian kernel of FWHM=$1\arcsec$ to maximize the contrast. We take a mesh size of $43\arcsec$ (doubles the size used in \citetalias{Greco2018}) to measure the local bacbkground and detect objects that are 0.7$\sigma$ per pixel above the local sky. We also require the object contains at least 100 contiguous pixels to further remove small compact objects. We perform the detection in $g$-band but require all the sources are also detected in $r$-band to exclude possible artifacts. 
    
    \item \textbf{Initial Sample Selection}. In this step, we take the output catalog from the previous step and remove those objects that are not likely to be LSBGs of interest. To be specific, we require objects have $g$-band half-light radius (measured by Source Extractor) larger than $r_{\rm min} = 2.0\arcsec$. We also require the measured colors to satisfy $-0.1 < g-i < 1.4$ and $|(g-r) - 0.7\cdot (g-i)| < 0.4$. %These cuts are the same as in \citetalias{Greco2018}. 
    %\textbf{Morphology? Auto-correlation? See `build_catalog.py` in `/tigress/jgreco/project-code/hugs-s18a-ana/scripts`}
    
    
\end{enumerate}
After these steps, we have an intial sample containing 86,002 LSBG candidates. \textbf{Why we obtain more sources per sqr deg compared with G18? Sky?} Among these objects, there are still a lot of false positives, including galaxy outskirts, tidal features, blended compact sources, etc. Therefore, we perform an extra ``deblending'' step to remove false positives.  


\subsection{Deblending}\label{sec:deblending}
A common type of false positive occurs when point-like sources are blended with diffuse lights from background galaxies or LSB outskirts of  stars and galaxies. In order to remove these objects from our sample, we perform non-parametric modeling for  each object using \code{scarlet}\footnote{\url{https://pmelchior.github.io/scarlet/}} \citep{Melchior2018}. \code{scarlet} is a deblending and modeling tool designed for multi-band data. It utilizes the color and morphology information to seperate blended sources. In the following, we briefly summarize how \code{scarlet} works, and we refer intersted readers to \citet{Melchior2018} for more details. In \code{scarlet}, each source in the scene $Y_i$ is described by a morphology matrix $S_i$ and a Spectral Energy Distribution (SED) vector $A_i$. The goal of modeling is to minimize the object function $f \propto \frac{1}{2}||\sum Y_i - P(\sum A_i S_i)||_2^{2}$ under certain constraints, where $P$ is the convolution with PSF. The morphology matrix of each source is limited to be within a bounding box. In this section, we assume that all sources have positive fluxes (positivity constraint) and the light profiles of all sources monotonically decreases from their centers to outskirts (monotonoicity constraint). Although not all galaxies satisfy the monotonicity constraint, it still provides an easy, sparse, and robust way to deblend overlapping sources. We refer to this modeling method as the ``vanilla scarlet''. 

In vanilla scarlet, objects can be modeled with different types of sources, including point source, single-extended source, multi-extended source, compact-extended source, and flat-sky source. The morphology matrix of point source is simply the normalized PSF. For the single-extended source, the morphology matrix follows positivity and monotonicity constraints. The multi-extended source is a combination of two or more single-extended sources, which makes it possible to model more complex structure and color gradient of galaxies. The compact-extended source is a single-extended source initialized using the morphology matrix of a point-source, which encourages the model to be compact. The flat-sky source has uniform color and morphology across the box.

Vanilla scarlet is ideal to identify and remove the false positives in our intial LSBG sample. 

We model LSBG candidates using vanilla scarlet as follows. 

\subsubsection{Peak detection}
We first generate cutout images with a size of $1\arcmin$ in $griz$-bands for each LSBG candiate. Then we construct a detection image by taking a weighted average of the images in four separate bands. We take the total inverse variance as the weight for each band. This detection image is considered to be deeper than any single band image. 

Next, we detect peaks on the detection image. We first run \code{sep} on the detection image using a detection threshold of 4$\sigma$, a mesh size of 48 pixels ($8\arcsec$), and a kernel size of 3 pixels. This step idenfities relatively extended sources in the detection image. However, there are still faint and compact peaks not detected. Therefore, we apply a wavelet decomposition to the detection image \citep{Starck2015} and only keep high spatial frequency components. Another round of \code{sep} is then ran on this high-frequency image using a detection threshold of 2.5$\sigma$, a mesh size of 24 pixels, and a kernel size of 3 pixels. This step detects many compact sources that are not detected in the previous step. In the end, we combine the two detection catalogs together. 

\subsubsection{Model initialization and optimization}
The deblending step is designed to deblend sources in the vicinity of the target object. The scene will contain both the target object and all the sources in its vicinity. Therefore, the sources being modeled is determined by how extended the target is. We estimate a bounding box based on the detection image, and include all the peaks inside this box to be modeled.

We use a multi-extended source with two components to model the target object, such that it is capable to capture the details of galaxy structure and color. The source is initialized as follows. We start by convolving the detection image with a circular Gaussian kernel with $\sigma=1.5$ pixels to boost the contrast between the signal and the sky noise \citep[e.g.,][]{Irwin1985,Akhlaghi2015,Greco2018}. After smoothing, the low surface brightness outskirts of target galaxy will become more prominent, which helps intializing the morphology matrix. The smoothed detection image is thresholded using $0.1\sigma$ to remove sky noise, then the morphology matrix $S_i$ of the intial model is determined by constructing a symmetric and monotonic approximation to the region in the detection image around the peak of the target object. The intial SED vector $A_i$ is set to be the color of the image averaged with the spatial weight function $S_i$. As a byproduct of intialization, we get a bounding box which represents the extent of the target object. We also add a flat-sky source to model the local sky around the target. This is helpful for situations where an object overlaps with the LSB outskirts of a bright galaxy.

Then we initialize all the sources inside the bounding box of the target. The extended objects detected in the first detection step are modeled as single-extended sources. For compact objects that are only detected in the second step, we model them as point sources if their FWHM $<$ 5 pixels, otherwise as compact-extended sources. The morphology matrices and SED vectors of these sources are also intialized in the same way as above. 

As mentinoed above, we only model sources within the bounding box of the target. However, scattered lights from nearby bright stars and galaxies could bias the modeling within the bounding box. We match our field with the GAIA catalog and mask out stars outside of the bounding box. Since we have already detected peaks throughout the whole cutout image, we also generate a mask for objects outside the bounding box to reduce the impact of scattered light from bright galaxies. 

As a result of such initialization, we only optimize the object function $f$ defined for sources within the bounding box of the target object. The optimization process uses the adaptive proximal gradient method \citep{Melchior2019}, which is a robust method for optimization with constraints. The model is considered to be converged when the relative changes of parameters are smaller than \code{e\_rel\,=\,2e-4}. Typically convergence is achieved after $\sim 50$ steps of opimization and the whole modeling takes about 40s for each LSBG. 

\textbf{For bright sources? Cookie cutter effect}

\begin{figure*}
	\vbox{ 
		%\vskip -10mm
		\centering
		\includegraphics[width=1\linewidth]{vanilla_scarlet_demo.pdf}
		\includegraphics[width=1\linewidth]{vanilla_scarlet_demo2.pdf}
	}
	\caption{An example of the deblending step.}
	\label{fig:vanilla_scarlet_demo}
\end{figure*}


\subsubsection{Measurement and false positive removal}\label{sec:non-par-measurement}
After running vanilla scarlet, we get non-parametric models of objects within the bounding box of the target. We take out the model of the target galaxy (as shown in Figure \ref{fig:vanilla_scarlet_demo}) and analyze it using \code{statmorph}\footnote{\url{https://statmorph.readthedocs.io/en/latest/}}\citep{statmorph}. The goal of this step is to remove false positives using various morphological and structural diagnostics.

\code{Statmorph} calculates non-parametric morphological and structural parameters, including the half-light radius, concentration-asymmetry-smoothness (CAS) statistics \citep{Conselice2003}, Gini-M20 \citep{Lotz2004}, etc. It takes an image, a segmentation map indicating the footprint of the object, a variance map, and PSF. For our purpose, the image is just the scarlet model convolved with the observed PSF. This image represents the deblended target galaxy. Because there is no contaminants in the deblended model, the segmentation map just covers the whole image. The variance map and PSF are taken from HSC cutouts. We also force the sky level to be zero because sky has already been fit in vanilla scarlet. 

One of the most important diagnostics is the size of the object. Half-light radius is calculated using ellpitical apertures, where the ellipticity $\varepsilon$ is determined by calculating the second moment of the image. We define the circularized half-light radius $r_e$ as the geometrical average of the semi-major and semi-minor axis of the half-light elliptical aperture: $r_e = r_{\rm sma} \sqrt{b/a}$. The total magnitudes are simply calculated by summing up the model in each band, and colors are defined by the difference among apparent magnitudes. Galactic extinction correction is not applied at this point. Two kinds of surface brightnesses are measured. The central surface brightness $\mu_0$ is measured by extrapolating the surface brightness profile to $r\to 0$. We also define the average surface brightness within effective radius as $\mu_{\rm eff} = m - 2.5 \log_{10}(2 \pi r_e^2)$, where $m$ is the total magnitude. 

We also use Gini-M20 and CAS statistics. The Gini coefficient \citep{Abraham2003,Lotz2004} and M20 statistics quantify how concentrated/extended is the flux distribution across the image. The CAS statistics characterize the concentration ($C = 5\log_{10}(r_{80} / r_{20}$), asymmetry, and smoothness of the light distribution of the object. We refer the readers to \citet{statmorph} for more details on their definitions and implementations. The morphological and structural parameters are the same in different bands by design of the \code{scarlet} model.


To better guide us on how to use these diagnostics, we visually inspected a subset of LSBG candiates in our initial sample (Sec \ref{sec:detection}) and use the labels to construct the metrics. We randomly selected 5,000 LSBG candidates that are matched with a host at $0.02 < z < 0.04$ (see Sec \ref{sec:match} for details). We run vanilla scarlet and measure morphological parameters for this subsample as described above. Then we generate color composite image for each object using HSC $griz$-bands with 0.5 arcmin on a side. The visual inspection was done by coauthors JEG, JG, SH, RB. Each object has been inspected by at least one person. Objects are classified into three types: \code{candy} (LSB galaxies with typicall dwarf galaxy features), \code{galaxy} (high surface brightness galaxies, or galaxies with spiral galaxy features), \code{junk} (false positives, including tidal streams, galaxy outskirts, and other artifacts). The difference between \code{candy} and \code{galaxy} can be quite ambiguous. In the end, we combine the votes from different persons as follows. An object is classified as \code{junk} if the number of votes as \code{junk} prevails the number of votes as \code{candy} and \code{galaxy}. For objects that are not \code{junk}, if the number of votes as \code{candy} is larger than that as \code{galaxy}, we assign this object as \code{candy}. Everything else is classified as \code{galaxy}. In total, there are 1146 \code{candies}, 2146 \code{galaxies}, and 1708 \code{junks}.

We use this visual inspection results to guide us on using the morphological and structural diagnostics. The distributions of \code{candy}, \code{galaxy}, and \code{junk} in the parameter space are shown in the top panels of Figure \ref{fig:deblending_cuts}. False positives (\code{junk}) are highlighted in red. Altough false positives are scattered all over the parameter space, a large fraction of them occupy regions with small size, extreme colors, very faint surface brightness, and large concentration and asymmetry. Therefore, we come up the following selection cuts to remove the false positives:
\begin{itemize}
    \item Color:
    \[0.0 < g-i < 1.2,\quad |(g-r) - 0.7\cdot (g-i)| < 0.25\]
    \item Size: \[1.8 \arcsec < r_e < 12 \arcsec\]
    \item Surface brightness: \[\mu_0(g) > 22.5,\quad 23.0 < \mu_{\rm eff}(g) < 27.5,\]
    \item Morphology: 
    \begin{gather*}
        \varepsilon < 0.65,\quad \mathrm{Gini} < 0.70,\quad M_{20} < -1.1,\\
        \mathrm{Gini} < -0.136\cdot M_{20} + 0.37,\\
        1.8 < C < 3.5,\quad A < 0.8.
    \end{gather*}
\end{itemize}
These criteria (shown as dashed lines) effectively help us remove objects that are not likely being LSBGs of interest. The color cuts here is narrower than the one used in detection to further remove junks and background high redshift galaxies. It is worth noticing that we not only remove objects with small sizes, but also remove objects with large size, low surface brightness, and high concentration. This is because the scarlet model of some junks can be very concentrated in the center but also extends quite far. The slant demarcation line on Gini-$M_{20}$ diagram is motivated by the line used in \citet{Lotz2008} to seperate merging galaxies and normal galaxies. 

The distributions of objects after applying the above cuts are shown in the bottom panels of Figure \ref{fig:deblending_cuts}. After the cuts, 97\% of the \code{junk} are removed, but the majority (70\%) of \code{candy} still remains. The number of \code{galaxy} also drops by 75\%. This empirical selection based on the non-parametric measurements on the scarlet models successfully removes most of the false positives and a large fraction of background galaxies in our initial sample. The numbers used in the selection are purely driven by the visual inspection results. 

It is worthy noticing that the scarlet modeling and non-parametric measurements is not designed to recover the intrinsic properties of the galaxies. They are only used as a diagnotic tool to remove false positives. Therefore, the half-light radius measured above does not correspond to the half-light radius of the galaxy itself, but merely a proxy. In reality, the vanilla scarlet model usually under-estimates the size and the total flux of LSB objects because the faint outskirts are typically not modeled well due to the non-parametric nature of vanilla scarlet and the monotonicity constraint. In Section \ref{sec:modeling}, we propose a method to better measure the size and total magnitude of LSBG using \code{scarlet}. We use the parameters measured in Section \ref{sec:modeling} for science. 

\begin{figure*}
	\vbox{ 
		%\vskip -10mm
		\centering
		\includegraphics[width=1.0\linewidth]{deblending_cuts.pdf}
	}
	\caption{Deblending cuts.}
	\label{fig:deblending_cuts}
\end{figure*}


Common false positives: why we want a deblending step

Why use scarlet: color info, non-par, monotonic, standard in future HSC, fast, high success rate. Vanilla scarlet.

We do not do scarlet fitting for the entire initial sample, because time consuming. We first match with MW hosts. 


\subsection{Modeling}\label{sec:modeling}
Although vanilla scarlet does a good job on removing false positives, the measured size, magnitude, and surface brightness on the vanilla scarlet model are quite biased. These measurements cannot be used to derive the physical properties of LSBGs. 

The non-parametric modeling in \code{scarlet} assumes weak correlation between pixels by imposing the monotonicity constraint. This leads to two problems when applying vanilla scarlet to the LSBGs. First, since pixels are not strongly correlated in the model, the model cannot capture the galaxy light at very low signal-to-noise ratio. Second, the monotonicity constraint stops the model to grow in certain directions if there is another source along this direction, otherwise the monotonicity is broken. This issue, dubbed as the ``cookie-cutter effect'', can be seen in Figure \ref{fig:vanilla_scarlet_demo}. As a result, the non-parametric model often does not capture the LSB outskirts of LSBG, thus biases the measuremenets. 

On the other side, parametric modeling imposes a very strong correlation on pixels. For example, using a 1-D light profile such as \sersic{}, pixels within an isophote are assumed to have the same intensity. Therefore, the signal-to-noise ratio in the outskirts is boosted by effectively combing the information from many pixels. The parametric model also does not suffer from the ``cookie-cutter'' issue. 

A traditional way of doing parametric modeling is to first mask out contaminants based on the detection segmentation map, then fit a model to the masked image. However, the fitting results are very sensitive to the masking scheme \citep[e.g.,][]{Greco2018}. A possible solution to this problem is to model all the objects in the cutout simultaneously using parametric models (e.g., in DECaLS, \citealt{Dey2019}). In this work, we combine the advantage of parametric modeling with the power of deblending in \code{scarlet}. To be specific, we follow the spirit of deblending as described in Section \ref{sec:deblending}, but replace the non-parametric model for the target with a parametric model. In this way, the LSB outskirts of LSBGs can be better captured with the parametric model and the impact of contaminants is minimized. 

We use the Spergel light profile to model the LSBGs. The Spergel profile is motivated by having a simple analytical expression in Fourier space, making it easy to be convolved with a PSF.\footnote{On the opposite, the \sersic{} profile does not have a simple analytic form in Fourier space.} The surface brightness of a Spergel profile \citep{Spergel2010} has a form of
\begin{equation}
    \label{eq:spergel}
    I_\nu(r) = \frac{c_{\nu}^{2} L_{0}}{2\pi r_{0}^{2}} f_{\nu}\left(\frac{c_{\nu} r}{r_{0}}\right),
\end{equation}
where 
\begin{equation}
    f_{\nu}(u)=\left(\frac{u}{2}\right)^{\nu} \frac{K_{\nu}(u)}{\Gamma(\nu+1)},
\end{equation}
and $K_\nu(u)$ is the Modified Bessel function of the second kind. The half-light radius is $r_0$, the total luminosity is $L_0$, and $c_\nu$ satisfyies the equation $(1 + \nu)f_{\nu + 1}(c_\nu) = 1/4$. Similar to the \sersic{} index, the parameter $\nu$ (Spergel index) controls the concentration of the light profile. As shown in Appendix \ref{ap:spergel}, the Spergel profile approximates the \sersic{} profile very well over the range of \sersic{} index that is interested to the study of LSBGs.

There are several changes in the modeling compared to the deblending step. First, we intialize a vanilla scarlet model for the target object as in Section \ref{sec:deblending}. Then we measure the half-light radius, total flux, and shape of the scarlet model. We use these numbers to initialize the Spergel profile. The size of the bounding box is also updated to be the maximum between 250 pixel and $10\, r_e$. Objects within this bounding box are included in the modeling, but only the target object is modeled by the Spergel profile. For the target, we still requires positivity constraint, as the mononicity is automatically satisfied. 

After optimization, we take the $r_0$ in Equation \eqref{eq:spergel} as the circularized half-light radius $r_e$, and take the $L_0$ as the total flux. The average surface brightness $\overline{\mu}_{\rm eff}$ is calculated in the same way as in Section \ref{sec:non-par-measurement}. 

We characterize the quality of the Spergel modeling by injecting mock \sersic{} galaxies into the cutout and compare the recovered properties with the truth, as shown in Appendix \ref{sec:meas_unc}. Overall speaking, the measurement agrees with the truth quite well. However, the size and the total flux of galaxies below $\overline{\mu}_{\rm eff} (g) > 27$ is under-estiamted. We also characterize the bias and the scatter in measurement as a function of size and surface brightness. We apply the bias correction to the measurement of LSBGs, and incorporate the measurement error into the science figures. 


\subsection{Completeness and Measurement Uncertainty}

\subsubsection{Completeness}\label{sec:completeness}
\begin{figure*}
	\vbox{ 
		%\vskip -10mm
		\centering
		\includegraphics[width=1\linewidth]{completeness.pdf}
	}
	\caption{Caption}
	\label{fig:completeness}
\end{figure*}

%Since the scarlet model is PSF-deconvolved, the color is not affected by the seeing differences among bands. 
After the source detection step and the deblending step, we obtain a sample of LSBG candidates where the fraction of false positives is very small. 

It is crucial to characterize the completeness of both detection and deblending in order to understand our survey efficiency and the properties of the LSBG population. We performed a large suite of image simulation to derive the completeness. The overall completeness is a combination of detection and deblending completenesses. The detection completeness is defined as the number of detected objects divided by the number of injected objects, whereas the deblending completeness is defined as the fraction of objects remained after cuts. 

For the detection completeness, we inject $\sim 35,000$ (check number with Johnny) mock galaxies with single \sersic{} light profiles \citep{Sersic1963} into the coadd images\footnote{We have also done extensive tests on injecting mock galaxies to the raw images and going through the entire data reduction pipeline. This is very expensive in terms of both CPU time and disk space, since we must run the full \code{hscPipe}. However, we find no noticable difference between this method and the direct injection to coadd images.}. The mock galaxies follow uniform distributions in size ($2\arcsec \leqslant r_{e} \leqslant 21\arcsec$), surface brightness ($23 \leqslant \overline{\mu}_{\rm eff}(g) \leqslant 28.5\ \mathrm{mag\ arcsec^{-2}}$), \sersic{} index ($0.8 < n < 1.2$), and ellipticity ($0 < \epsilon < 0.6$), resembling the LSBG distribution in \citetalias{Greco2018}. They are randomly assigned to have a blue ($g-i=0.47,\ g-r=0.32$) and red ($g-i=0.82,\ g-r=0.56$) color with equal chance.

We find the detection completeness mainly depends the size and surface brightness of mock galaxy. As shown in the left panel of Figure \ref{fig:completeness}, the detection completeness remains high across different sizes. It drops below 50\% when the surface brightness gets fainter than 27.5 mag arcsec$^{-2}$. We find negligible dependendce of completeness on \sersic{} index, color, and additional structure of galaxy (such as star-forming clumps). However, we find the completeness declines above ellipticity $\varepsilon > 0.6$, suggesting that edge-on disk galaxies may be missing from our sample. In this work, we neglect the dependences of detection completeness on parameters other than size and surface brightness. 

\textbf{Need to update the completeness to the S18A version. Ping Johnny.}

For the deblending completeness, we inject 5,000 mock \sersic{} galaxies into the coadd and run vanilla scarlet on them. The mock galaxies follow the same uniform distribution in size, surface brightness, ellipticity, \sersic{} index as for deriving the detection completeness, but follow a Gaussian distribution in color: $g-i \sim \mathcal{N}(0.6, 0.2^2),\ g-r = 0.7 (g-i) + \mathcal{N}(0, 0.03^2)$. Following the procedure described in Sec \ref{sec:deblending}, we measure the non-parametric sizes and surface brightnesses on the scarlet models of the mock galaxies. The deblending completeness is shown in the middel panel of Figure \ref{fig:completeness}. The deblending completeness is high at bright surface brightnesses, but starts to decline with increasing size and surface brightness. Mock galaxies fainter than $\overline{\mu}_{\rm eff}(g) > 27.5$ are mostly removed by the deblending step, likely due to the blending of other small compact sources with the mock galaxy. The deblending step is very effective on removing false positives, but at a cost of losing very LSB galaxies. 

The combined completeness is shown in the right panel of Figure \ref{fig:completeness}. The countours shows the 70\%, 50\%, and 20\% completenesses, respectively. \textbf{Also compare our completeness with other surveys.}

%\textbf{Make sure that completeness is assigned based on the bias-corrected size and SB.} Checked. No problem.

\subsubsection{Measurement uncertainty}\label{sec:meas_unc}
We characterize the quality of the Spergel modeling by injecting mock \sersic{} galaxies into the cutout and compare the recovered properties with the truth, as shown in Appendix \ref{sec:meas_unc}. Overall speaking, the measurement agrees with the truth quite well. However, the size and the total flux of galaxies below $\overline{\mu}_{\rm eff} (g) > 27\ \sbunit$ is under-estiamted. We also characterize the bias and the scatter in measurement as a function of size and surface brightness. We apply the bias correction to the measurement of LSBGs, and incorporate the measurement error into the science figures. 

In order to test how well do we recover the photometric and structural parameters in the modeling step (Sec \ref{sec:modeling}), we take the 5,000 mock \sersic{} galaxies used for computing the deblending completeness and model them using the Spergel light profile. We compare the fitting results with the ground truth in Figure \ref{fig:meas_bias}, where $\Delta(X) = X(\rm truth) - X(\rm measured)$, and x-axis corresponds to the truth values. We find that the measured half-light radius $r_e$ is biased to be smaller than the truth, and the bias depends on the surface brightness and apparent size. For surface brightness fainter than $27\ \sbunit$, the measured size can be much smaller than the truth. As a result, the apparent magnitude $m$ and the average surface brightness $\overline{\mu}_{\rm eff}$ are also measured to be fainter than the truth. 

To correct for this bias in size, surface brightness, apparant magnitude, and color, we assume that the bias only depends on the size and surface brightness. Indeed, we do not find any significant dependence of the bias on color, ellpticity, and Spergel index. We split the $r_e-\overline{\mu}_{\rm eff}(g)$ plane using a $8\times 8$ grid, and calculate the mean bias within each bin. Then we interpolate over the grid using a multiquadratic kernel\footnote{\url{https://docs.scipy.org/doc/scipy/reference/generated/scipy.interpolate.RBFInterpolator.html}}. Figure \ref{fig:meas_err} shows the interpolated bias terms as a function of the measured size and surface brightness. For half-light radius, we take $(r_{e, \rm t} - r_{e, \rm m}) / r_{e,\rm m}$ as the bias term; for surface brightness, we take $\overline{\mu}_{\rm eff, t}(g) - \overline{\mu}_{\rm eff, m}(g)$ as the bias term. As shown in Figure \ref{fig:meas_err}, the size and surface brightness bias grows as increasing surface brightness. The bias has a peak at $r_e\sim 6\arcsec$ because the $r_e$ shown in the figure is the measured size, not the truth size. It is the bias in size measurement that makes galaxies with large intrinsic $r_e$ pile up around $r_e\sim 6\arcsec$. We also find that the bias in $g-i$ color is quite small. 

For LSBGs in our sample, we apply corrections for the bias using the interpolated bias terms. We first correct for the bias in size, $g$-band average surface brightness, $g-r$ and $g-i$ colors. Then we calculate the $g$-band apparent magnitude following $m_g = \overline{\mu}_{\rm eff}(g) - 2.5\log(2\pi r_e^2)$. The magnitudes and surface brightnesses in other bands are derived using $g$-band magnitude, surface brightness, and colors. In this way, we apply a self-consistent correction for the measurement bias to the data. 

Besides, it is also important to characterize the measurement uncertainty. The measurement uncertainty consists of a statistical uncertainty which is determined by the shape of the likelihood (posterior) surface, and a systematic uncertainty which is related to various factors including sky subtraction, contamination effects, etc. Unlike other parametric modeling code such as \code{galfit} or \code{imfit}, \code{scarlet} does not explore the full likelihood (posterior) space but rather finds one optimal solution. Thus we cannot get a statistical error of the measured properties from \code{scarlet}. However, by comparing the recovered properties of mock galaxies with the truth, we can empirically estimate the measurement uncertainty without knowing the impact of each factor. Following the same way of calculating the bias, in each bin, we compute the standard deviation of the difference between the truth and the bias-corrected measurement, then we interpolate over the grid. The measurement errors are shown in Figure \ref{fig:meas_err} as contours. 


%All satellite photometry is in the AB system and cor- rected for MW dust extinction using the E(B − V ) maps of Schlegel et al. (1998) recalibrated by Schlafly & Finkbeiner (2011). We take the solar g-band magni- tude to be Mg⊙ = 5.03 mag (Willmer 2018).

TBD: re-think the way of computing the error, especially for $r_e$. 


vdB 16 doesn't consider whether GALFIT gives the corrrect $R_e$, when deriving the completeness (recovered fraction)

Remember to ref SMUDGES papers here.

\begin{figure*}
	\vbox{ 
		\centering
		\includegraphics[width=1\linewidth]{meas_bias.pdf}
	}
    \caption{Caption}
    \label{fig:meas_bias}
\end{figure*}


\begin{figure*}
	\vbox{ 
		\centering
		\includegraphics[width=1\linewidth]{meas_error_spergel.pdf}
	}
    \caption{Caption}
    \label{fig:meas_err}
\end{figure*}

\section{UDGs in Milky-Way analogs}

\subsection{Matching with Milky-Way analogs}\label{sec:match}
The goal of this paper is to study the UDG population hosted by MW-like galaxies. However, the properties of MW itself vary in literature \citep{Licquia2015,Bland-Hawthorn2016}, and the definitions of MW analogs are also different among groups. In the SAGA survey \citep{SAGA-I,SAGA-II}, MW analogs are selected based on their absolute $K$-band magnitude $-23 > M_K > -24.6$, which is derived using abundance matching by assuming a simple galaxy-halo connection model (Fig 2 of \citealt{SAGA-I}). This luminosity range approximately corresponds to a stellar mass range of $10.2 < \log\, M_\star/M_\odot < 11.0$. They also require the MW analogs to be in isolation (without nearby bright galaxies) and lie in a redshift range of $0.005 < z < 0.01$ (20-40 Mpc). In the ELVES survey \citep{ELVES-I,ELVES-II,CarlstenELVES2022}, the requirements for MW-like host is loosened to be $M_K < -22.1$ ($M_\star > 10^{9.9}\ M_\odot$) because the probed volume by ELVES ($D<12$ Mpc) is smaller than that of SAGA. We choose the stellar mass range of MW analogs to be $10.2 < \log\, M_\star/M_\odot < 11.2$, which is simply a 1 dex bin centered at the measured stellar mass of the Milky Way ($\sim 10^{10.7}\ M_\odot$, \citealt{Licquia2015}). MW analogs selected using this definition is very close to those in SAGA but are slightly more massive than the ELVES hosts.

Since UDGs are relatively scarce in MW-like hosts \citep{SAGA-II,CarlstenELVES2022}, it is helpful to probe a larger volume to obtain good statistics. However, UDGs will be too small and faint to be detected in HSC images beyond certain distances. We choose our redshift range to be $0.01 < z < 0.04$, which makes sure that we can detect a significant number of faint dwarf galaxies around MW-like hosts. We exclude galaxies $z<0.01$ because 1) the number is very small; 2) their large angular size makes them shredded in deblending step, including them will introduce a lot of spurious LSB objects. Our MW analogs sample complements the ELVES sample and SAGA sample in redshift range. 

After applying the stellar mass and redshift cuts to the NSA catalog, there are 23,218 galaxies left. Then we match them to the LSBG catalog (described in Section \ref{sec:detection}) as follows. For a given MW-like host, we first calculate its virial radius $R_{\rm vir}$ assuming the stellar mass-halo mass relation in \citet{Behroozi2010}. It turns out that 40\% of hosts have virial radii larger than 300 kpc. Then we identify any LSBG that falls into the projected angular virial radius of the host. If one LSBG is matched to multiple hosts, we assign it to the nearest host based on the separation normalized by host virial radius. Finally, we have 901 MW-like hosts and 10,579 LSBG candidates associated with them. However, there are still a significant fraction of spurious objects in this LSBG sample, including galactic cirrus, tidal tails/streams, shredded large galaxies, compact sources in galaxy outskirts, etc. As introduced in Section \ref{sec:deblending}, we perform a deblending step to effectively remove these spurious objects. There are 2,673 objects left after the deblending cuts. Next, we model them using the Spergel profile as described in Section \ref{sec:modeling}. Based on the modeling results, we apply the same color cuts as in Section \ref{sec:deblending} and a size cut of $1.6\arcsec < r_e < 15\arcsec$ to further remove unreliable fits and false positives. We also remove duplicated objects. In the end, we have 2510 LSBG candidates as our final LSBG sample, from which we construct the ultra-diffuse galaxy (UDG) sample and the ultra-puffy galaxy (UPG) sample as follows. 

\subsection{UDG and UPG sample}\label{sec:sample}
The UDG population is usually defined based on the surface brightness and the physical size of the galaxy. However, there are many different criteria in literature: \citep{vanDokkum2015} define UDGs to have effective radius $r_e$ larger than 1.5 kpc and central surface brightness $\mu_0(g)$ fainter than $24.0\ \sbunit$; other groups also use the surface brightness at $r_e$ \citep[e.g.,][]{DiCintio2017,Cardona-Barrero2020} or the average surface brightness within $r_e$ to define UDGs \citep[e.g.,][]{Koda2015,Yagi2016,vdBurg2016,Leisman2017,Martin2019}, and the size criterion also varies from $>1$ kpc to $>1.5$ kpc. \citet{vanNest2022} explore these definitions and find that different choice of UDG could drastically affect which subset of dwarfs are selected, therefore affect our understanding of the UDG population.

The measured central surface brightness might be biased by nuclear star clusters \citep{Neumayer2020} or other contaminants. Therefore in this work, we use the average surface brightness within effective radius $\sbeff$ to define UDG. The difference between the average surface brightness and the central surface brightness for \sersic{} profile can be analytically calculated, where $\overline{\mu}_{\mathrm{eff}} - \mu_0 = 1.124$ for $n=1$ and 0.796 for $n=0.8$ \citep{Graham2005,Yagi2016}. Since dwarf galaxies and UDGs typically has \sersic{} index in a range of $0.8 < n < 1.2$ \citep{vanDokkum2015,ELVES-I}, we take $\overline{\mu}_{\mathrm{eff}} - \mu_0 = 1$ as an average value. In this work, we define UDGs to be galaxies with $r_e+\sigma(r_e) > 1.5$ kpc and $\sbeff + \sigma(\sbeff) > 25\ \sbunit$, which takes the measurement errors into account. This definition maximizes the consistency with the definition in \citet{vanDokkum2015} while not losing UDGs harboring nuclear star clusters.

Among the 2510 LSBG candidates, there are 432 objects satisfying the UDG definition. We did a final visual inspection for these objects and excluded 16 objects that are very likely to be false positives. We also removed another 4 objects having completeness less than 0.1. In the end, we obtain our UDG sample with 412 objects, which are associated with 258 hosts. The total sky area occupied by UDG hosts (out to 1 $R_{\rm{vir}}$ is 32.71 deg$^{2}$. The UDG catalog is available \textbf{online} and the catalog format is shown in Table \ref{tab:catalog}. 

\vspace{1em}

The concept of UDG was proposed for those dwarf galaxies with large physical size and diffuse light distribution. However, the size of a galaxy is strongly correlated with its mass, as known as the mass-size relation \citep[e.g.,][]{Lange2015}. The slope of the mass-size relation has been shown to be color-dependent for galaxies above $M_\star > 10^{8.5}\ M_\odot$: blue star-forming galaxies have a shallower slope, whereas red quenched galaxies have a steeper slope. However, recent works show that the mass-size relation in the dwarf galaxy regime ($10^{5.5} < M_\star < 10^{8.5}$) is quite universal: the slope and intercept are not sensitive to the color and morphology of dwarf galaxy \citep{ELVES-I}. This gives us a chance to define a subset of dwarf galaxies that are outliers on the mass-size plane. Taking the measured mass-size relation from \citet{ELVES-I} $\log\, (r_e/\mathrm{pc}) = 0.247\, \log\, (M_\star/M_\odot) + 1.071$ and a scatter of $\sigma=0.181$ dex, we define a population of ultra-puffy galaxies (UPGs) to be galaxies that are $>1.5\sigma$ above the mass-size relation. The UPGs are not necessarily large in physical size or faint in average surface brightness. Both UDG and UPG represent subsets of dwarf galaxies that are larger and diffuser than average. However, the hard cut on size (1.5 kpc) makes UDG losing ``large'' galaxies at the low-mass end because as mass goes smaller, the average size also gets smaller. The surface brightness cut of UDG also biases the UDG sample in color distribution and further bias the quenched fraction, as will be discussed in Section \ref{sec:quench}. The definition of UPG based on the observed mass-size relation is more physics-motivated, and UPG better represents a homogeneous subset of dwarf galaxies along the stellar mass axis. There is also no color preference in the UPG definition for a given stellar mass. It is worth noticing that definition based on mass-size relation was already used in simulation studies where simulations cannot reproduce the observed mass-size relation \citep[e.g.,][]{Benavides2021}.

There are 362 objects in our LSBG candidates satisfying that they are $1.5\sigma$ above the ridge line of the mass-size relation. After visual inspection and removing objects with completeness less than 0.1, we have 337 galaxies in our UPG sample. These UPGs are associated with 239 hosts. The total sky area occupied by UPG hosts (out to 1 $R_{\rm{vir}}$) is 32.37 deg$^{2}$.


The distribution of the UDG and UPG sample on the size-surface brightness plane are shown in Figure \ref{fig:udg_upg_re_mu}. The galaxies are split into two color bins and are shown in blue ($g-i < 0.8$) and red ($g-i > 0.8$), respectively. Two marginal plots show the unnormalized histograms of galaxies in the two color bins. The numbers of red and blue galaxies are similar in the UDG sample, but there are more blue galaxies than red ones in the UPG sample. Since there is no hard surface brightness cut, the UPG sample includes a lot of blue galaxies with higher surface brightness ($\sbeff < 25$). The UPG sample also lose a significant fraction of red galaxies at $25 < \sbeff < 26$, because they are not LSB (hence low-mass) enough to be higher than $1.5\sigma$ from the mass-size relation.

\begin{figure*}
	\vbox{ 
		\centering
		\includegraphics[width=1\linewidth]{udg_upg_sample.pdf}
	}
    \caption{The distribution of galaxies in the ultra-diffuse galaxy (UDG) sample (\textit{left}) and the ultra-puffy galaxy (UPG) sample (\textit{right}). The UDGs are galaxies with $r_e>1.5$ kpc and $\sbeff > 25.0\ \sbunit$. The UPGs is defined to be $1.5\sigma$ above the mass-size relation in \citet{ELVES-I}. We split the samples into two color bins and show them in red ($g-i>0.8$) and blue ($g-i<0.8$). The marginal histograms are not normalized to highlight the relative number of red and blue galaxies. Compared with the UDG sample, the UPG sample includes blue galaxies with surface brightness higher than the UDG cut ($\sbeff < 25\ \sbunit$) and excludes red galaxies at $25 < \sbeff < 26\ \sbunit$.
    }
    \label{fig:udg_upg_re_mu}
\end{figure*}

We derive the stellar masses of galaxies in the UDG and UPG sample from the Spergel model fitting and a color-$M_{\star}/L$ relation from \citet{Into2013}:
\begin{align*}
&\log \left(M_{\star} / L_{g}\right)=1.774\,(g-r)-0.783, \\
&\log \left(M_{\star} / L_{g}\right)=1.297\,(g-i)-0.855.
\end{align*}
Because we have both $g-r$ and $g-i$ color available from the model fitting, we use the average of the $M_{\star}/L$ derived from the two colors for calculating stellar mass. We assume the solar absolute magnitude in $g$-band to be 5.03 \citep{Willmer2018}. We corrected for the effect of Galactic extinction on colors based on \citet{SFD1998,Schlafly2011}. The magnitudes and surface brightnesses are also extinction-corrected. 


\subsection{Background contamination}\label{sec:bkg}
The major weakness of LSBG searches in photometric data is that distance information is typically unknown. Although we have matched LSBGs to MW analogs with known redshifts, the affiliation of LSBGs to the host is still not guaranteed. As a result, a certain fraction of galaxies in our UDG and UPG sample might just be foreground or background galaxies that fall within the virial radius of the host by projection. Given the volume probed by our search, contamination is more probable to be dominated by background galaxies. We estimate the fraction of contamination as follows.

We selected a patch of sky with 24 deg$^{2}$ in HSC S18A. Then we performed the same deblending and modeling steps for 2707 LSBGs detected in this region. \footnote{The sky region is randomly selected, and these LSBGs share the same properties with our initial LSBG sample described in Section \ref{sec:match}.} We also removed objects that are already in our UDG sample, and we did visual inspection to further remove spurious objects. In the end, there are 480 LSBGs representing a population of possible contaminants in our UDG sample. Next, because the UDG is defined based on a physical size, we randomly associate the 483 LSBGs to the hosts of galaxies in our UDG sample. Then we calculate a physical radius for each LSBG and classify an LSBG as a ``fake`` UDG if it satisfies the definition. We repeat the random matching for 200 times, and we obtain 11,458 fake UDGs. In this way, the number density of fake UDGs is estimated to be $S_{\rm UDG} = 1.94\pm0.03\ \mathrm{deg}^{-2}$. In the same way, we get 11,867 fake UPGs in 200 random matches, and the estimated number density of fake UPGs is $S_{\rm UPG} = 2.12\pm0.03\ \mathrm{deg}^{-2}$. The catalogs of fake UDG and UPGs are also available publicly\footnote{\url{https://zenodo.org/}}.

Using the number densities and the probed area of our survey, we calculate the fraction of contamination for both UDG and UPG sample. For the UDG sample, there are $f_{\rm contam} \approx 16\%$ contaminants; for the UPG sample, the contamination fraction is $f_{\rm contam} \approx 20\%$. We take the contamination fraction into account when calculating the average UDG (UPG) number in Section \ref{sec:results}. 

One of the main goals of this paper is to study the star formation status of UDGs and UPGs in MW analogs. The star formation rate of galaxies is indicated by broad-band colors, and we use color to calculate the quenched fraction of UDGs and UPGs. However, background contaminants affect the color distribution of the UDG (UPG) sample, thus affect the calculated quenched fraction. If the fake UDGs (UPGs) span a different region from the true UDGs in the observable space, we can use these observables to assign weights to each object in our UDG sample indicating the possibility of being a true UDG. Indeed, we find that the $g-i$ color distribution of the fake UDG sample is bluer than that of the true UDG sample, which motivates us to assign an importance weights based on $g-i$ color. We first compute the normalized histograms of $g-i$ colors (denoted as $\lambda_k$ at $(g-i)_{k}$ bins) for both true UDG and fake UDG samples. Then we multiply $f_{\rm contam}$ with the histogram of fake UDG sample ($\lambda_k^{\rm fake}$) such that the histogram sums up to the average contamination fraction. The weights of UDGs in color bin $(g-i)_k$ is estimated to be $w_k = \max\,(1 - f_{\rm contam} \lambda_k^{\rm fake} / \lambda_k^{\rm true}, 0)$. This weight stands for a possibility of being a contaminant. The weights of UPGs are assigned following the same procedure. We find this color-based contamination subtraction scheme is sufficient for this work since the fraction of contamination is small. Our main results shown in Section \ref{sec:results} are robust against contamination subtraction. 


\section{Results}\label{sec:results}
In this section, we present statistical analysis of UDGs and UPGs associated with MW analogs in the nearby Universe. TLDR the full section here?

\subsection{Abundance of UDGs and UPGs}\label{sec:n_udg}

\begin{figure*}
	\vbox{ 
		\centering
		\includegraphics[width=1\linewidth]{N_UDG_host_mass.pdf}
	}
    \caption{The abundance of UDGs as a function of host halo mass. \textit{Left}: We compile the UDG abundance measurements from literature covering a wide range of host mass. The UDG abundance of our work $N_{\rm UDG} = 0.76\pm 0.04$ is shown as red square, the power-law relation from \citet{vdBurg2017} is shown in gray, and our fitting result is shown in pink. Our UDG abundance is consistent with ELVES but significantly higher than that in SAGA. After including more measurements at the lower halo mass end, the power-law is shallower than the relation reported in \citet{vdBurg2017}. \textit{Right}: We split the UDG sample into bins based on the host halo mass (triangles) and $g-i$ color (pentagons). The UDG abundance is higher for more massive host and redder host. }
    \label{fig:n_udg}
\end{figure*}

As demonstrated by many previous studies \citep[e.g.,][]{vdBurg2016,vdBurg2017}, the average number of UDGs per host scales with host halo mass. While many literature focus on finding UDGs in clusters and large groups, there are less constraints on the UDG abundance at the lower halo mass end. In this section, we calculate the UDG abundance of MW analogs, and compare with other surveys.

We define the UDG abundance as the average number of UDGs per host galaxy. In our UDG sample, there are 412 UDGs associated with 258 hosts. After correcting for background contamination (see Section \ref{sec:bkg}) and completeness (see Section \ref{sec:completeness}), the UDG abundance of MW analogs is $N_{\rm UDG} = 0.76\pm 0.04$ per host. Here we neglect the fact that the number of satellites contained within the virial sphere is different from the number of satellite within a projected cylinder with virial radius (the so-called deprojection factor, \citealt{vdBurg2017}).

We compare our UDG abundance with other surveys focused on MW analogs. In the SAGA survey \citep{SAGA-II}, 6 satellite galaxies (out of 127 spectroscopy-confirmed satellites around 36 MW analogs) satisfy the definition of UDG. Therefore the UDG fraction in \citet{SAGA-II} is about $N_{\rm UDG,\, SAGA} \approx 0.17\pm0.07$. In the ELVES survey \citep{CarlstenELVES2022}, there are 15 satellites (out of 404 satellites associated with 30 MW analogs) satisfying the UDG definition, leading to a UDG fraction of $N_{\rm UDG,\, ELVES} \approx 0.50\pm0.13$. \citet{Roman2017b} identified 11 UDGs around three galaxy groups in the IAC Stripe 82 Legacy Survey \citep{Fliri2016}. Since both SAGA and ELVES select satellites based on distance measurements and do not correct for completeness, their UDG abundances are interpreted as lower limits. \citet{Roman2017b} do not estimate the background contamination fraction and do not correct for completeness either. We plot the UDG abundances of these surveys in the left panel of Figure \ref{fig:n_udg}, together with the results for larger galaxy groups and clusters \citep{Koda2015,Munoz2015,Roman2017a,Roman2017b,Janssens2017,vdBurg2017}. Among these studies, only \citet{vdBurg2017} applied background subtraction and completeness correction to the raw counts. The power-law regression result from \citet{vdBurg2017} $N_{\rm UDG} \propto M_h^{1.11\pm 0.07}$ is shown in gray. The UDG abundance of this work is highlighted as the red square. As shown in Figure \ref{fig:n_udg}, the scatter of $N_{\rm UDG}-M_h$ relation get larger at the lower halo mass end. Our UDG abundance is similar to the value in ELVES survey, but is higher than the prediction from \citet{vdBurg2017} but is still considered consistent within scatter. The UDG abundances from SAGA is significantly lower than both ELVES and our result. This might be explained by different photometric depth of data used. SAGA selects dwarf candidates in DECaLS data, which is x mag shallower than the CFTH and HSC data used in ELVES and this work. It is probable that SAGA missed a significant fraction of red and lower surface brightness galaxies. 

Taking all data points from literature (as shown in Figure \ref{fig:n_udg}), we use orthogonal distance regression (ODR) to fit a power-law between $N_{\rm UDG}$ and host halo mass $M_h$ and find $N_{\rm UDG} \propto M_h^{0.96\pm 0.04}$, shown as the pink dashed line in Figure \ref{fig:n_udg}. The slope of the power law is shallower than \citet{vdBurg2017} ($\beta=1.11\pm0.07$) but still steeper than \citet{Roman2017b} ($\beta=0.85\pm0.05$). Since most individual studies do not have a statistical estimation for background contamination and completeness, the UDG abundances for large groups should be interpreted as lower limits. If completeness is applied, it is possible that the slope $\beta$ will be steeper than what we are showing in Figure \ref{fig:n_udg}.

In the right panel, we split the UDG sample into different bins based on their host halo mass (shown as triangles) and color (shown as pentagons). The UDG abundance is slightly higher for hosts with higher halo mass and redder color, but the trend is stronger on host color. 


% Plot the UDG and UPG population on the re-abs mag plane (Bovy book)
% \citet{Trujillo2020}: define size based on surface mass density
% We correct for completeness after statistical contamination subtraction.

\subsection{Size distribution}
\begin{figure*}
	\vbox{ 
		\centering
		\includegraphics[width=1\linewidth]{size_distribution.pdf}
	}
    \caption{Caption}
    \label{fig:size_distribution}
\end{figure*}

The size distribution of UDGs is an important topic since it has been used to test various UDG formation scenario \citep[e.g.,][]{Amorisco2016,vdBurg2017}. In this section, we calculate the size distribution of our UDG and UPG sample. We split UDGs into 10 bins in $\log r_e$ with a bin width of 0.06 dex. Then we calculate the number of UDGs per bin . 


We also take the measurement error in $r_e$ into account. 

\subsection{Spatial distribution}



\begin{figure*}
	\vbox{ 
		\centering
		\includegraphics[width=1\linewidth]{radial_distribution.pdf}
	}
    \caption{Caption}
    \label{fig:radial_distribution}
\end{figure*}


\subsection{Quenched fraction}\label{sec:quench}

\citep{Baxter2021}

\citep{Akins2021}: quenched fraction and quenching timescale in DC Justice League simulation.\citep{Samuel2022}: quenched fraction of MW sats in FIRE-2.

When discussing the quenched fraction, also mention that we are uncertain about the mass-size above 8.5 and below 9-ish? 


This also explains the low quenched fraction of SAGA \citep{CarlstenELVES2022}.


\begin{figure*}
	\vbox{ 
		\centering
		\includegraphics[width=1\linewidth]{mass_size_plane.pdf}
	}
    \caption{Caption}
    \label{fig:mass_size}
\end{figure*}

\begin{figure*}
	\vbox{ 
		\centering
		\includegraphics[width=1\linewidth]{quenched_frac_dist2host.pdf}
	}
    \caption{Caption}
    \label{fig:qfrac_dist2host}
\end{figure*}

\begin{figure*}
	\vbox{ 
		\centering
		\includegraphics[width=1\linewidth]{quenched_frac_host_color.pdf}
	}
    \caption{Caption}
    \label{fig:qfrac_host_color}
\end{figure*}

\section{Discussion and Conclusion}
Scenarios:
1. violent environmental effects: radial orbits, back-splash both explain large size and quenching. But according to Bernavidis et al, the quenched fraction drops with increasing mass. 

2. accreting field UDGs, and quench them due to ram pressure stripping. But how long do accreted UDGs last before being destroyed? Need to look at ram pressure stripping literature.

Homework:
3. Move the stellar mass bin definition, try to match with red/blue, spiral/elliptical plot. If cannot match, does this hint the formation scenarios?
4. How significant do we detect blue UDGs? 
5. Split into redshift bins


UDG rejuvenate after ram pressure stripping?

Why red is equivalent to quenched? Need to clarify in the paper

calculate probability of being a backsplash satellite (e.g., splashed to 1.5 Mpc)

Compare spatial distribution with normal dwarf? iF they are indeed high-spin tail of normal dwarfs. 

Don't forget to discuss \citep{xSAGA-I}


\section*{Acknowledgment}
JL is grateful for discussions with XXX. The authors thank Yao-Yuan Mao for his tool for visualizing image cutouts. 

The Hyper Suprime-Cam (HSC) collaboration includes the astronomical communities of Japan and Taiwan, and Princeton University. The HSC instrumentation and software were developed by National Astronomical Observatory of Japan (NAOJ), Kavli Institute for the Physics and Mathematics of the Universe (Kavli IPMU), University of Tokyo, High Energy Accelerator Research Organization (KEK), Academia Sinica Institute for Astronomy and Astrophysics in Taiwan (ASIAA) and Princeton University.  
Funding was contributed by the FIRST program from Japanese Cabinet Office, Ministry of Education, Culture, Sports, Science and Technology (MEXT), Japan Society for the Promotion of Science (JSPS), Japan Science and Technology Agency (JST), Toray Science Foundation, NAOJ, Kavli IPMU, KEK, ASIAA and Princeton University.

The authors are pleased to acknowledge that the work reported on in this paper was substantially performed using the Princeton Research Computing resources at Princeton University which is consortium of groups led by the Princeton Institute for Computational Science and Engineering (PICSciE) and Office of Information Technology's Research Computing.

\vspace{1em}
\software{\href{http://www.numpy.org}{\code{NumPy}} \citep{Numpy},
          \href{https://www.astropy.org/}{\code{Astropy}} \citep{astropy}, \href{https://www.scipy.org}{\code{SciPy}} \citep{scipy}, \href{https://matplotlib.org}{\code{Matplotlib}} \citep{matplotlib},
          \href{https://halotools.readthedocs.io/en/latest}{\code{Halotools}} \citep{Hearin2017},
          \href{https://pmelchior.github.io/scarlet/}{\code{scarlet}} \citep{Melchior2018}, \href{https://github.com/dr-guangtou/unagi}{\code{unagi}}.
          }


\bibliography{citation}{}
\bibliographystyle{aasjournal}



\newpage
\appendix 

\section{Spergel profiles}\label{ap:spergel}

In this appendix, we demonstrate that the Spergel profile can approximate \sersic{} profile and profile a lookup table for the correspondence between \sersic{} index $n$ and Spergel index $\nu$.

The surface brightness of a \sersic{} profile follows \citep{Sersic1963,Graham2005}:
\begin{equation}\label{eq:sersic}
    I(r)=I_{\mathrm{e}} \exp \left\{-b_{n}\left[\left(\frac{r}{r_{\mathrm{e}}}\right)^{1 / n}-1\right]\right\},
\end{equation}
where $r_e$ is the half-light radius, $I_e$ is the surface brightness at $r=r_e$, $n$ is the \sersic{} index. The value of $b_n$ satisfies $\Gamma(2 n)=2 \gamma\left(2 n, b_{n}\right)$, where $\gamma(a, x)$ is the incomplete gamma function. According to \citet{Graham2005}, the total luminosity of a \sersic{} profile is given by 
\begin{equation}\label{eq:sersic_lum}
    L_0 = I_{e} r_{e}^{2}\, 2 \pi n\, e^{b_{n}} \left(b_{n}\right)^{-2 n} \Gamma(2 n).
\end{equation}
The Spergel profile is described in Section \ref{sec:modeling}. 

To study the correspondence between \sersic{} and Spergel profiles, we generate \sersic{} profiles with different \sersic{} indices, and try to fit Spergel profiles to \sersic{} ones. The \sersic{} index ranges from $n=0.5$ to $n=4.5$, and both profiles are normalized using $r_e$. For each \sersic{} profile, we calculate the total luminosity $L_0$ according to \eqref{eq:sersic_lum} and plug it into the Spergel profile \eqref{eq:spergel} as a fixed value. Therefore, only the Spergel index $\nu$ is allowed to vary during the fitting. The best-fit Spergel index $\nu$ as a function of \sersic{} index $n$ is shown in the left panel of Figure \ref{fig:spgl_calib}. As two examples, we show two \sersic{} profiles (solid) and their best-fit Spergel profiles (dash-dotted) in the right panel. A Spergel profile with $\nu=0.5$ is exactly an exponential profile with $n=1$. The de vaucouleurs profile \citep{deVaucouleurs1948} with $n=4$ can be approximated by a Spergel profile with $\nu=-0.74$. 

As shown in Figure \ref{fig:spgl_calib}, a \sersic{} profile with small \sersic{} index ($0.5 < n < 1.5$) can be well-approximated by a Spergel profile, although the Spergel profile seems to be more extended than \sersic{} in the outskirts. For a \sersic{} profile with high \sersic{} index, the approximation gets worse at both small and large radii. Overall, the Spergel profile is a good approximation to \sersic{} from $n\approx 0.5$ to $n\approx 4.5$. It is well-known that the light profiles of low-mass galaxies are quite flat and can be described using \sersic{} profiles with $0.5 < n < 1.5$ \citep[e.g.,][]{Lange2015,Greco2018,ELVES-I}. It is thus reasonalbe to use Spergel profiles to model the LSBGs (Sec \ref{sec:modeling}) and enjoy its convenience in the Fourier space. 

\begin{figure*}
	\vbox{ 
		\centering
		\includegraphics[width=0.75\linewidth]{spergel_sersic_calib.pdf}
	}
    \caption{The correspondence between the \sersic{} profile \eqref{eq:sersic} and the Spergel profile \eqref{eq:spergel}. We fit a Spergel profile to \sersic{} while fixing the total luminosity and half-light radius. The left panel shows the best-fit Spergel index $\nu$ as a function of \sersic{} index $n$. In the right panel, two \sersic{} profiles (solid) and their best-fit Spergel profiles (dash-dotted) are shown. Spergel profile approximates \sersic{} well for small \sersic{} indices.  
    }
    \label{fig:spgl_calib}
\end{figure*}


\section{Measurement Error}\label{ap:meas_error}

We characterize the quality of the Spergel modeling by injecting mock \sersic{} galaxies into the cutout and compare the recovered properties with the truth, as shown in Appendix \ref{ap:meas_error}. Overall speaking, the measurement agrees with the truth quite well. However, the size and the total flux of galaxies below $\overline{\mu}_{\rm eff} (g) > 27$ is under-estiamted. We also characterize the bias and the scatter in measurement as a function of size and surface brightness. We apply the bias correction to the measurement of LSBGs, and incorporate the measurement error into the science figures. 



vdB 16 doesn't consider whether GALFIT gives the corrrect $R_e$, when deriving the completeness (recovered fraction)

Remember to ref SMUDGES papers here.


\section{UDG and UPG Catalogs}
\onecolumngrid 

\begin{table}
\caption{UDG (UPG) catalog description} 
\label{tab:catalog}
\begin{center}
\begin{tabular}{l l l}
\hline\hline
% \multicolumn{3}{c}{Table: LSBGs (781 rows)}                 \\
% \hline
Column Name      & Unit    & Description                    \\
\hline
ID                       &         & Unique LSBG ID \\
ra                       & deg     & Right ascension (J2000) \\
dec                      & deg     & Declination (J2000) \\
$r_e$         & arcsec  & Circularized effective radius  \\
$\sigma(r_e)$ & arcsec  & Uncertainty of $r_e$ \\
$\overline{\mu}_{\mathrm{eff}}(g)$               & $\sbunit$ & $g$-band average surface brightness within $r_e$ \\
$\sigma(\overline{\mu}_{\mathrm{eff}}(g))$       & $\sbunit$ & Uncertainty of $\overline{\mu}_{\mathrm{eff}}(g)$           \\
$\overline{\mu}_{\mathrm{eff}}(r)$               & $\sbunit$ & $r$-band average surface brightness within $r_e$ \\
$\sigma(\overline{\mu}_{\mathrm{eff}}(r))$       & $\sbunit$ & Uncertainty of $\overline{\mu}_{\mathrm{eff}}(r)$           \\
$\overline{\mu}_{\mathrm{eff}}(i)$               & $\sbunit$ & $i$-band average surface brightness within $r_e$ \\
$\sigma(\overline{\mu}_{\mathrm{eff}}(i))$       & $\sbunit$ & Uncertainty of $\overline{\mu}_{\mathrm{eff}}(i)$           \\
$m_g$                    & mag     & $g$-band apparent magnitude     \\
$\sigma(m_g)$            & mag     & Uncertainty of $m_g$            \\
$m_r$                    & mag     & $r$-band apparent magnitude     \\
$\sigma(m_r)$            & mag     & Uncertainty of $m_r$            \\
$m_i$                    & mag     & $i$-band apparent magnitude     \\
$\sigma(m_i)$            & mag     & Uncertainty of $m_i$            \\
$g-r$                    & mag     & $g-r$ color                     \\
$\sigma(g-r)$            & mag     & Uncertainty of $g-r$ color      \\
$g-i$                    & mag     & $g-i$ color                     \\
$\sigma(g-i)$            & mag     & Uncertainty of $g-i$ color      \\
$\nu$                    &         & Spergel index              \\
$\varepsilon$            &         & Ellipticity                     \\
$A_g$                    & mag     & $g$-band Galactic extinction \\
$A_r$                    & mag     & $r$-band Galactic extinction \\
$A_i$                    & mag     & $i$-band Galactic extinction \\
$\log\ M_\star$ & $M_\odot$ & Stellar mass of UDG (UPG) \\
comp & & Completeness \\
weight & & Color-dependent weight (see Section \ref{sec:bkg}) \\
host\_name & & Name of the host galaxy \\
host\_ra & & Right ascension (J2000) of the host galaxy \\
host\_dec & & Declination (J2000) of the host galaxy \\
host\_log\_m\_star & $M_\odot$ & Host stellar mass\\
host\_r\_vir & kpc & Virial radius of the host galaxy \\
host\_g\_i & mag & $g-i$ color of the host galaxy \\
host\_z &  & Redshift of the hsot galaxy \\
sep\_to\_host & deg & Angular separation between the host and UDG (UPG)\\
\hline\hline
\end{tabular}
\end{center}
\tablecomments{
These tables are published in their entirety in machine-readable format.
Magnitudes are on the AB system and have not been corrected for Galactic
extinction. The information of the host galaxies is from NASA-Sloan Atlas. We provide Galactic extinction corrections, which are derived from
the \citet{Schlafly2011} recalibration of the \citet{SFD1998} dust maps. 
}
\end{table}

Also add external link for fake UDG catalog, etc. 


\end{CJK*}
\end{document}