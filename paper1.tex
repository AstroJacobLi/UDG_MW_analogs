\documentclass[twocolumn,astrosymb,twocolappendix]{aastex631}

%% The default is a single spaced, 10 point font, single spaced article.
%% There are 5 other style options available via an optional argument. They
%% can be invoked like this:
%%
%% \documentclass[arguments]{aastex631}
%% 
%% where the layout options are:
%%
%%  twocolumn   : two text columns, 10 point font, single spaced article.
%%                This is the most compact and represent the final published
%%                derived PDF copy of the accepted manuscript from the publisher
%%  manuscript  : one text column, 12 point font, double spaced article.
%%  preprint    : one text column, 12 point font, single spaced article.  
%%  preprint2   : two text columns, 12 point font, single spaced article.
%%  modern      : a stylish, single text column, 12 point font, article with
%% 		  wider left and right margins. This uses the Daniel
%% 		  Foreman-Mackey and David Hogg design.
%%  RNAAS       : Supresses an abstract. Originally for RNAAS manuscripts 
%%                but now that abstracts are required this is obsolete for
%%                AAS Journals. Authors might need it for other reasons. DO NOT
%%                use \begin{abstract} and \end{abstract} with this style.
%%
%% Note that you can submit to the AAS Journals in any of these 6 styles.
%%
%% There are other optional arguments one can invoke to allow other stylistic
%% actions. The available options are:
%%
%%   astrosymb    : Loads Astrosymb font and define \astrocommands. 
%%   tighten      : Makes baselineskip slightly smaller, only works with 
%%                  the twocolumn substyle.
%%   times        : uses times font instead of the default
%%   linenumbers  : turn on lineno package.
%%   trackchanges : required to see the revision mark up and print its output
%%   longauthor   : Do not use the more compressed footnote style (default) for 
%%                  the author/collaboration/affiliations. Instead print all
%%                  affiliation information after each name. Creates a much 
%%                  longer author list but may be desirable for short 
%%                  author papers.
%% twocolappendix : make 2 column appendix.
%%   anonymous    : Do not show the authors, affiliations and acknowledgments 
%%                  for dual anonymous review.
%%
%% these can be used in any combination, e.g.
%%
%% \documentclass[twocolumn,linenumbers,trackchanges]{aastex631}
%%
%% AASTeX v6.* now includes \hyperref support. While we have built in specific
%% defaults into the classfile you can manually override them with the
%% \hypersetup command. For example,
%%
%% \hypersetup{linkcolor=red,citecolor=green,filecolor=cyan,urlcolor=magenta}
%%
%% will change the color of the internal links to red, the links to the
%% bibliography to green, the file links to cyan, and the external links to
%% magenta. Additional information on \hyperref options can be found here:
%% https://www.tug.org/applications/hyperref/manual.html#x1-40003
%%
%% Note that in v6.3 "bookmarks" has been changed to "true" in hyperref
%% to improve the accessibility of the compiled pdf file.
%%
%% If you want to create your own macros, you can do so
%% using \newcommand. Your macros should appear before
%% the \begin{document} command.
%%
\newcommand{\vdag}{(v)^\dagger}
\newcommand\aastex{AAS\TeX}
\newcommand\latex{La\TeX}
% \newcommand{\sbunit}{mag~arcsec$^{-2}$}
\newcommand{\sbunit}{\mathrm{mag\ arcsec}^{-2}}
\newcommand{\sbcen}{\mu_{0}(g)}
\newcommand{\sbeff}{\overline{\mu}_{\mathrm{eff}}(g)}
\newcommand{\sbeffr}{\overline{\mu}_{\mathrm{eff}}(r)}
\newcommand{\jiaxuan}[1]{\textcolor{orange}{\textbf{Jiaxuan: #1}}}


% \newcommand{\code}[1]{\textbf{\texttt{#1}}}
\newcommand{\code}[1]{\texttt{#1}}
\newcommand{\sersic}{S\'ersic}
\usepackage{CJKutf8}
\usepackage{bm}
\usepackage{appendix}
\usepackage{amsmath,amssymb}

%% Reintroduced the \received and \accepted commands from AASTeX v5.2
% \received{\today}
% \revised{\today}
% \accepted{\today}

% \submitjournal{ApJ}

%% alias for citations
\defcitealias{Greco2018}{G18}

\shorttitle{Mass-size outliers in MW satellites}
\shortauthors{Li et al.}
%%
%% You can add a light gray and diagonal water-mark to the first page 
%% with this command:
%% \watermark{text}
%% where "text", e.g. DRAFT, is the text to appear.  If the text is 
%% long you can control the water-mark size with:
%% \setwatermarkfontsize{dimension}
%% where dimension is any recognized LaTeX dimension, e.g. pt, in, etc.
%%
%%%%%%%%%%%%%%%%%%%%%%%%%%%%%%%%%%%%%%%%%%%%%%%%%%%%%%%%%%%%%%%%%%%%%%%%%%%%%%%%
\graphicspath{{./}{figures/}}
%% This is the end of the preamble.  Indicate the beginning of the
%% manuscript itself with \begin{document}.

\begin{document}
\begin{CJK*}{UTF8}{gbsn}


\title{Beyond Ultra-Diffuse Galaxies I: Mass-Size Outliers in the Satellites of Milky Way Analogs}

\correspondingauthor{Jiaxuan Li}
\author[0000-0001-9592-4190]{Jiaxuan Li (李嘉轩)}
\affiliation{Department of Astrophysical Sciences, 4 Ivy Lane, Princeton University, Princeton, NJ 08544, USA}
\email{jiaxuanl@princeton.edu}

\author[0000-0002-5612-3427]{Jenny E. Greene}
\affiliation{Department of Astrophysical Sciences, 4 Ivy Lane, Princeton University, Princeton, NJ 08544, USA}

\author[0000-0003-4970-2874]{Johnny Greco}
\affiliation{Department of Astrophysical Sciences, 4 Ivy Lane, Princeton University, Princeton, NJ 08544, USA}
\affiliation{Center for Cosmology and AstroParticle Physics (CCAPP), The Ohio State University, Columbus, OH 43210, USA}

\author[0000-0003-1385-7591]{Song Huang (黄崧)}
\affiliation{Department of Astrophysical Sciences, 4 Ivy Lane, Princeton University, Princeton, NJ 08544, USA}
\affiliation{Department of Astronomy and Tsinghua Center for Astrophysics, Tsinghua University, Beijing 100084, China}

\author[0000-0002-8873-5065]{Peter Melchior}
\affiliation{Department of Astrophysical Sciences, 4 Ivy Lane, Princeton University, Princeton, NJ 08544, USA}
\affiliation{Center for Statistics \& Machine Learning, Princeton University, Princeton, NJ 08544, USA}

\author[0000-0002-1841-2252]{Rachael Beaton}
\affiliation{Department of Astrophysical Sciences, 4 Ivy Lane, Princeton University, Princeton, NJ 08544, USA}
\author[0000-0002-2991-9251]{Kirsten Casey}
\affiliation{Center for Cosmology and AstroParticle Physics (CCAPP), The Ohio State University, Columbus, OH 43210, USA}
\author[0000-0002-1841-2252]{Shany Danieli}
\affiliation{Department of Astrophysical Sciences, 4 Ivy Lane, Princeton University, Princeton, NJ 08544, USA}
\author[0000-0002-1841-2252]{Andy Goulding}
\affiliation{Department of Astrophysical Sciences, 4 Ivy Lane, Princeton University, Princeton, NJ 08544, USA}
\author[0000-0002-2704-5028]{Remy Joseph}
\affiliation{Department of Astrophysical Sciences, 4 Ivy Lane, Princeton University, Princeton, NJ 08544, USA}
\affiliation{Oskar Klein Centre for Cosmoparticle Physics, Department of Physics, Stockholm University, Stockholm SE-106 91, Sweden}
\author[0000-0002-1841-2252]{Erin Kado-Fong}
\affiliation{Department of Astrophysical Sciences, 4 Ivy Lane, Princeton University, Princeton, NJ 08544, USA}

\author[0000-0002-1418-3309]{Ji Hoon Kim}
\affiliation{Astronomy Program, Department of Physics and Astronomy, Seoul National University, 1 Gwanak-ro, Gwanak-gu, Seoul 08826, Republic of Korea}
\affiliation{SNU Astronomy Research Center, Seoul National University, 1 Gwanak-ro, Gwanak-gu, Seoul 08826, Republic of Korea}
\author{Lauren MacArthur}
\affiliation{Department of Astrophysical Sciences, 4 Ivy Lane, Princeton University, Princeton, NJ 08544, USA}

%% Note that the \and command from previous versions of AASTeX is now
%% depreciated in this version as it is no longer necessary. AASTeX 
%% automatically takes care of all commas and "and"s between authors names.

%% AASTeX 6.31 has the new \collaboration and \nocollaboration commands to
%% provide the collaboration status of a group of authors. These commands 
%% can be used either before or after the list of corresponding authors. The
%% argument for \collaboration is the collaboration identifier. Authors are
%% encouraged to surround collaboration identifiers with ()s. The 
%% \nocollaboration command takes no argument and exists to indicate that
%% the nearby authors are not part of surrounding collaborations.

%% Mark off the abstract in the ``abstract'' environment. 
\begin{abstract}
Large diffuse galaxies are hard to find, but understanding the environments where they live and their numbers, and ultimately their origins, is of intense interest and important for galaxy formation and evolution. Using the Hyper Suprime-Cam data, we perform a systematic search for low surface-brightness galaxies and present novel and effective methods for detecting and modeling them. 
As a case study, we build a statistical sample of galaxies that are outliers in the mass-size relation and are satellites of Milky Way analogs in the nearby Universe ($0.01 < z < 0.04$). These ``ultra-puffy'' galaxies (UPGs), defined to be $1.5\sigma$ above the average mass-size relation, better represents the tail of satellite size distribution. We find that each MW analog hosts $N_{\rm UPG} = 0.61\pm 0.04$ ultra-puffy galaxies on average, which is comparable to the observed abundance at this halo mass in the Local Volume. We also construct a sample of ultra-diffuse galaxies (UDGs) in MW analogs and find an abundance of $N_{\rm UDG} = 0.76\pm0.04$ per host. We compare with literature results including UDG searches in clusters and massive groups. The UDG abundance scales with the host halo mass following a sublinear power law. We argue that our definition for ultra-puffy galaxies, which is based on the mass-size relation, is more physically-motivated and objective than the common definition of ultra-diffuse galaxies, which depends on a surface brightness cut and thus yields different surface mass density cuts for quenched and star-forming galaxies. 
\end{abstract}

\keywords{Low surface brightness galaxies (940), Dwarf galaxies (416), Galaxy properties (615), Galaxy abundances (574)}


\section{Introduction} \label{sec:intro}
In a variety of galaxies in the Universe, an interesting subset of galaxies with low surface brightness and large physical size have been identified \citep[dubbed low surface brightness galaxies or LSBGs, e.g.,][]{Sandage1984,Caldwell1987,Impey1988,McGaugh1995,Dalcanton1997a}. Recently, new excitement has been generated by the discovery of a large population of ultra-diffuse galaxies (UDGs) in the Coma cluster \citep{vanDokkum2015}, accompanied by many other works on detecting UDGs in clusters \citep[e.g.,][]{Koda2015,Mihos2015,Yagi2016,vdBurg2016,vdBurg2017,Lee2017,ManceraPina2018,Zaritsky2019}, groups \citep[e.g.,][]{Roman2017b,Greco2018,SAGA-II,CarlstenELVES2022}, and the field \citep[e.g.,][]{Leisman2017,Roman2019,Prole2019,Tanoglidis2021,Kadowaki2021}. There have been a variety of studies of these diffuse galaxy systems, ranging from their dark matter content \citep[e.g.,][]{Mowla2017,vanDokkum2018,vanDokkum2019,Wasserman2019,Keim2022}, to their globular cluster populations \citep[e.g.,][]{vanDokkum2017,Somalwar2020,Forbes2020,Danieli2022,Gannon2022,vanDokkum2022GC}, and their stellar populations \citep[e.g.,][]{Gu2018,Ferre-Mateu2018,Villaume2022}. UDGs are defined to be galaxies with large size ($r_e > 1.5$ kpc) and low surface brightness ($\sbcen > 24.0\ \sbunit$), and thus have very large sizes for their stellar mass compared to normal dwarf galaxies. Therefore, a central question about LSBGs, including UDGs, is what causes them to be outliers from the average mass-size relation?
% Therefore, a central question about mass-size outliers including UDGs is what physical mechanisms are responsible for the formation and quenching of these large diffuse galaxies.

Various mechanisms have been proposed to explain the presence of large, diffuse galaxies (as mass-size outliers) across different environments. In the field (i.e, in isolation), their large sizes might be caused by stellar feedback \citep{DiCintio2017,Chan2018}, early galaxy mergers \citep{Wright2021}, passing into and out of a more massive halo (so-called backsplash satellites, \citealt{Benavides2021}), or by inhabiting halos with higher spin \citep{Dalcanton1997,Amorisco2016,Liao2019,Benavides2022}. In groups and clusters, their large sizes may ensue from tidal heating near pericenter \citep{Jiang2019}, via adiabatic expansion due to mass loss \citep{Tremmel2020}, or even via a bullet-dwarf collision \citep{vandokkum2022Nat,vanDokkum2022GC}. Despite theoretical advances, it is still an open question how mass-size outliers like UDGs are formed and quenched. It is also unclear whether mass-size outliers belong to a distinct class of galaxies or are just a natural extension of dwarf systems to lower surface brightness or larger size. A nice way to to answer these questions is to compare these mass-size outliers with ``normal'' low mass galaxies for which we have a better understanding. 


Our best understanding of the low mass galaxy regime comes from the satellite system of our Milky Way (MW) and the Local Group (LG) \citep[e.g.,][]{McConnachie2012,Simon2019}. With the advent of deep sky surveys and dedicated spectroscopic programs, we are now in a position to also map the satellite systems of MW analogs in the Local Volume and nearby Universe. The Satellites Around Galactic Analogs (SAGA) survey is designed to spectroscopically confirm the satellites of 100 MW analogs at 20-40 Mpc \citep{SAGA-I,SAGA-II}. The Exploration of Local VolumE Satellites (ELVES,  \citealt{ELVES-I,ELVES-II,CarlstenELVES2022}) survey uses deep imaging data to identify satellite candidates of MW-like hosts in the Local Volume (LV, $D<12$ Mpc) and estimate surface brightness fluctuation distances from images to confirm their association with their hosts. Having these MW satellites as a reference frame, we can contrast them with the large diffuse galaxies in the same environment to understand the formation of diffuse galaxies. However, there are only $\sim 40$ confirmed UDGs associated with MW analogs in the literature \citep{Roman2017b,Cohen2018,SAGA-II,CarlstenELVES2022,Karunakaran2022b}. A larger sample of mass-size outliers (including UDGs) associated with MW-like hosts is needed to conduct such a comparison between the mass-size outliers and the normal satellites.

The goal of this paper is to find and study large diffuse galaxies in the context of the satellite systems of MW analogs. Leveraging the depth and wide sky coverage of Hyper-Suprime Cam (HSC) Subaru Strategic Program imaging data, we perform a search for LSBGs using novel methods in detection, false positive identification, and modeling. We further build a large sample of mass-size outliers that are likely satellites of MW-mass hosts in the nearby Universe. We propose a new definition for mass-size outliers, dubbed ``ultra-puffy galaxies'' (UPGs), based on the observed mass-size relation in the Local Volume using ELVES sample. This concept is less affected by the galaxy color and provides a robust avenue to study the mass-size outlier population. 
We examine the abundance of mass-size outliers in MW analogs and compare with their abundances in other environments. Our results seek to shed light on the role of environment on the formation of large diffuse galaxies. 
%By comparing mass-size outliers with normal satellites and simulations, our results seek to shed light on the formation and evolution of the large diffuse galaxies. 

% The occurrence rate of UDGs in these low-density environments is unknown, and whether UDGs are quenched in such a regime has not been explored. With a sample of UDGs around MW-like hosts, we can compare the structural parameters, quenched fractions, nucleation fractions, and radial distributions with normal satellites in the nearby universe \citep[e.g.,][]{SAGA-II,CarlstenELVES2022}, to shed light on questions about the formation and evolution of the UDGs.

The layout of this paper is as follows. Section \ref{sec:data} describes the data used in this work. In Section \ref{sec:lsbg_search}, we describe the LSBG search in the HSC data, including source detection (\S \ref{sec:detection}), deblending (\S \ref{sec:deblending}), modeling (\S \ref{sec:modeling}), and completeness and uncertainty (\S\ref{sec:comp_meas}). In Section \ref{sec:sample_construction}, we propose a new definition for large and diffuse galaxies (the UPGs) and present our UDG and UPG samples. In Section \ref{sec:results}, we show statistical analysis for UDGs and UPGs including their distribution on the mass-size plane and their abundances (\S \ref{sec:n_udg}). We further argue why we need to move beyond UDG and adopt UPG definition in Section \ref{sec:discussion}. Section \ref{sec:summary} presents a summary of this work and prospects for the future. In a companion paper (Li et al., to be submitted), we discuss the size distributions, spatial distributions, and star formation quenching of UDGs and UPGs. 

In this work, we refer to LSBGs as objects selected purely by surface brightness, whereas mass-size outliers (including UDGs and UPGs) are selected based on physical size, surface brightness, or stellar mass. We adopt a flat $\Lambda$CDM cosmology from \citet{Planck15} with $\Omega_{\rm m}= 0.307$ and $H_0 = 67.7\ $km s$^{-1}$ Mpc$^{-1}$. We use the AB system \citep{Oke1983} for magnitudes. The stellar mass used in this work is based on a \citet{Chabrier2003} initial mass function.


\section{Data} \label{sec:data}
\subsection{Hyper Suprime-Cam Subaru Strategic Program}
The Hyper Suprime-Cam Subaru Strategic Program (HSC-SSP, \citealt{Aihara2018}; hereafter the HSC survey)\footnote{\url{https://hsc-release.mtk.nao.ac.jp/doc/}} is an optical imaging survey using the 8.2-m Subaru telescope and the Hyper Suprime-Camera \citep{Miyazaki2012, Miyazaki2018}. The \texttt{Wide} layer is designed to cover $\sim 1000\ \rm{deg}^{2}$ of the sky in five broad bands ($grizy$), reaching a depth of $g=26.6$ mag, $r=26.2$ mag and $i=26.2$ mag ($5\sigma$ point source detection). HSC data are processed using the \code{hscPipe}\footnote{\url{https://hsc.mtk.nao.ac.jp/pipedoc_e/}} \citep{Bosch2018}, which is a customized version of the Legacy Survey of Space and Time (LSST) pipeline \citep{LSST-pipeline}\footnote{\url{https://pipelines.lsst.io/}}.

In this work, we use the \code{Wide} layer coadd data from the Public Data Release 2 (PDR2, also known as \code{S18A}, \citealt{Aihara2018}) of the HSC survey. It covers $\sim 300\ \rm{deg}^2$ in all five bands, which is 1.5 times larger than the dataset (\code{S16A}) analyzed in \citet{Greco2018}. One of the key improvements made in PDR2 is the sky background subtraction. Compared with previous data releases, PDR2 adopted a full focal plane sky subtraction algorithm to overcome the over-subtraction of the local sky background around bright objects \citep{Aihara2018,Li2021}. The unprecedented depth and careful sky subtraction make PDR2 an ideal dataset to study low surface brightness galaxies. In this work we use the PSF models generated by \code{hscPipe}. 


\subsection{NASA-Sloan Atlas}
We use the NASA-Sloan Atlas (NSA\footnote{\url{http://nsatlas.org}}, \citealt{Blanton2005,Blanton2011}) to select galaxies that are analogous to our Milky Way based on their stellar mass (see \S\ref{sec:match}), then we match the LSBGs to these MW analogs. The NSA catalog provides various physical properties of galaxies in the nearby Universe as derived from the Sloan Digital Sky Survey \citep[SDSS,][]{York2000}. We use the most recent version of the NSA catalog (\code{v1\_0\_1}\footnote{\url{https://www.sdss.org/dr13/manga/manga-target-selection/nsa/}}) which contains about $640,000$ galaxies out to $z < 0.15$. This version also updates the aperture photometry to elliptical Petrosian photometry since it is considered to be more reliable than the photometry used in the older versions. In this paper, we use the stellar mass derived from elliptical Petrosian photometry using \code{kcorrect v4\_2} \citep{Blanton2007}. The redshifts of galaxies in NSA come from several spectroscopic surveys, \ion{H}{1} gas surveys, or direct distance measurements. 

\section{LSBG search in HSC PDR2}\label{sec:lsbg_search}

Continuing the work in \citet{Greco2018} (hereafter \citetalias{Greco2018}), we conduct a systematic search for low surface brightness galaxies (LSBGs) in the HSC PDR2 data, which covers $\sim 1.5$ times the sky area of \citetalias{Greco2018}. The schematic of the searching method in this paper remains similar to that in \citetalias{Greco2018}, but we do improve the searching completeness and purity by adding several new steps, which are highlighted in this section. We outline the major steps below and refer the reader who are already familiar with the algorithms in LSBG searches to \S \ref{sec:sample_construction} for UDG and UPG sample construction.

\begin{enumerate}
    \item Source Detection (\S \ref{sec:detection}): we run \href{https://www.astromatic.net/software/sextractor/}{\code{SourceExtractor}} \citep{Bertin1996} on the coadd images after removing bright extended sources. Then we apply an initial size and color cut based on the output catalog. 
    \item Deblending (\S \ref{sec:deblending}): we remove false positives that are not likely to be LSBGs by running \href{https://pmelchior.github.io/scarlet/}{\code{scarlet}} \citep{Melchior2018}. We use the \code{scarlet} models to define color-size-morphology-surface brightness cuts. This step removes roughly 97\% of false positives.
    \item Modeling (\S \ref{sec:modeling}): we fit a parametric model to the LSBGs to estimate their structural parameters. 
    \item Completeness and measurement uncertainties (\S \ref{sec:comp_meas}) are characterized by injecting mock \sersic{} galaxies into images. 
\end{enumerate}
The number of objects remained after each step is shown in Table \ref{tab:steps_flow}. Our source detection pipeline \code{hugs}\footnote{\url{https://github.com/johnnygreco/hugs}} and deblending and modeling code \code{kuaizi}\footnote{\url{https://github.com/AstroJacobLi/kuaizi}} are open-sourced and available online.

\begin{deluxetable*}{lcc}
\tablecaption{LSBG search steps and the number of remaining objects after each step}
\tablewidth{20cm}
\label{tab:steps_flow}
\tablehead{
\colhead{Process} & \colhead{Description} &
\colhead{Remaining objects}
}
\startdata
Initial detection & \S\ref{sec:detection} & 86,002 \\
Matching with MW analogs at $0.01 < z < 0.04$ & \S\ref{sec:match} & 10,579 \\
Deblending &\S\ref{sec:deblending} & 2,673\\
Spergel profile modeling & \S\ref{sec:modeling}, \S\ref{sec:match} & 2,510\\
Completeness & \S\ref{sec:comp_meas},\S\ref{ap:comp_meas_unc} & 2,510 \\
Measurement error and uncertainty & \S\ref{sec:comp_meas},\S\ref{ap:comp_meas_unc} & 2,510\\
Mass-size outlier selection & \S\ref{sec:sample} & 362 (UPG), 432 (UDG)
\enddata
\end{deluxetable*}

\subsection{Source Detection}\label{sec:detection}
\citetalias{Greco2018} performed a search for LSBGs in the first $\sim 200$ deg$^2$ of the HSC survey and uncovered 781 LSBGs. We continue the work in \citetalias{Greco2018} and extend the search to the HSC PDR2 data, which covers $\sim 300\ \rm{deg}^{2}$ and has much better sky subtraction for LSB studies compared to previous data releases. We follow the same method for source detection as in \citetalias{Greco2018}, but make several updates to accommodate PDR2 data. These updates are guided by both our understanding of PDR2 data and completeness tests using mock galaxies. We summarize the main steps of LSBG search method here, and refer the interested readers to \citetalias{Greco2018} and Appendix \ref{ap:detection} for more details. 
% In the following, we summarize the main steps of the search and emphasize the updates made to improve the overall completeness and purity. We refer the interested readers to \citetalias{Greco2018} for more details.

We start by replacing the bright sources and their associated LSB outskirts with the sky noise. Bright objects and their diffuse light are identified using surface brightness thresholding. Then we run a detection using \href{https://sep.readthedocs.io/en/v1.1.x}{\code{sep}} \citep{Barbary2016} to remove small compact sources and noisy peaks and also replace their footprints with sky noise. We use \code{SourceExtractor} to detect sources on the ``cleaned'' images with a low detection threshold. Taking the output catalog from \code{SourceExtractor}, we apply an initial selection based on the size, color, and peakiness of objects. 

After these steps, we have an initial sample that contains 86,002 LSBG candidates. We obtain four times more sources per square degree ($\sim 280\ \mathrm{deg}^{-2}$) than \citetalias{Greco2018} mainly because HSC PDR2 has a sky subtraction suited for finding LSB features, our detection here is more inclusive of larger objects, and our size cut is less restrictive for small objects. Among these objects, there are still a lot of false positives, including shredded galaxy outskirts, tidal features, Galactic cirrus, and blended compact sources. Therefore, we perform an extra ``deblending'' step to further remove false positives.  

% {\bf Here you put one PP summarizing the main changes, and then put the rest of the details in 3.1 into Appendix 1. End this section with the numbers of targets at the end of this step, and I agree with Erin that having a flow-table with the numbers per step would be nice.}

\subsection{Deblending}\label{sec:deblending}

\begin{figure*}
	\vbox{ 
		%\vskip -10mm
		\centering
		\includegraphics[width=1\linewidth]{vanilla_scarlet_demo.pdf}
		\includegraphics[width=1\linewidth]{vanilla_scarlet_demo2.pdf}
	}
	\caption{A demonstration of the deblending step as described in \S\ref{sec:deblending} and \S\ref{ap:deblending}. Here we show two objects from our initial sample: the one shown in the top panels is a blue LSBG, whereas the one shown in the bottom is a false positive detection which is a blend between high-$z$ galaxy and galaxy outskirts. The first columns show the $griz$ color-composite images with the bounding boxes overlaid. The objects covered by the gray shades are masked during fitting. The remaining columns show the initial model, optimized models (PSF convolved), residual image, and the optimized model of the target galaxy only. 
	}
	\label{fig:vanilla_scarlet_demo}
\end{figure*}

Removing false positives from the initial sample while retaining high completeness is one of the major challenges in LSBG searches \citep[e.g.,][]{vanDokkum2015,Koda2015,Yagi2016,Greco2018,SAGA-I,Zaritsky2019,Zaritsky2021,Tanoglidis2021,Zaritsky2022}. A common type of false positive occurs when point-like sources are blended with diffuse light from background galaxies or the LSB outskirts of stars and galaxies. Our initial detection cannot distinguish them from real LSBGs. In order to identify and remove these objects from our sample, we perform non-parametric modeling using \href{https://pmelchior.github.io/scarlet/}{\code{scarlet}} \citep{Melchior2018} and design an efficient metric to remove false positives.


\subsubsection{\code{Scarlet} fitting}
\code{Scarlet} is a deblending and modeling tool designed for multi-band and multi-resolution imaging data. It utilizes color and morphology information to separate blended objects. In the following, we briefly summarize how \code{scarlet} works, and we refer the interested readers to \citet{Melchior2018,Melchior2021} and the online documentation\footnote{\url{https://pmelchior.github.io/scarlet/}} for more details. 

In \code{scarlet}, each source $k$ in the cutout is described by a morphology image $S_k$ and a Spectral Energy Distribution (SED) vector $A_k$, and the multi-band images is $Y$. The goal of the modeling is to minimize the objective function $f = \frac{1}{2} ||Y - P \ast (\sum_k A_k^\top \times S_k)||^{2}$ under certain constraints, where $P$ is the PSF and $*$ is convolution. The morphology image of each source is limited by a bounding box. We assume two constraints when running \code{scarlet}: all sources have positive fluxes (positivity constraint) and the light profiles of all sources monotonically decrease from center to outskirts (monotonoicity constraint). The monotonicity constraint is over-simplified for well-resolved galaxies with complicated structures, but for objects in our sample this assumption still holds for most cases and provides an effective and robust way to deblend overlapping sources. We refer to this modeling method as \textit{vanilla \code{scarlet}}. 

% Similar to many other modeling codes (e.g., \code{galfit} \citealt{galfit}, \code{imfit} \citealt{imfit}), \code{scarlet} is initialized with the positions of detected peaks in the prior steps. The models are then optimized subject to the constraints. Using the \code{scarlet} model, we extract structural and morphological parameters and use them to exclude false positives from our LSBG sample. In the following subsections, we describe our peak detection, model initialization, and measurement procedures. 

We use vanilla \code{scarlet} to model the LSBG candidate and extract structural and morphological parameters from the \code{scarlet} model, which are further used to exclude false positives from our LSBG sample. In the following subsections, we briefly describe how we detect peaks on the image, initialize and optimize \code{scarlet} models. We refer the readers to Appendix \ref{ap:deblending} for more detailed descriptions. 

The deblending step is designed to model the sources in the vicinity of the LSBG candidate. Therefore, we run \code{sep} on the $griz$-combined image to detect objects (i.e., peaks) around the LSBG candidate. Objects that are close to the LSBG candidate are modeled using \code{scarlet} models. The models are initialized based on the smoothed image to capture the LSB outskirts of the target galaxy. Then the models are optimized using the adaptive proximal gradient method \citep{Melchior2019}. Typically, convergence is achieved after $\sim 50$ steps of optimization and the whole modeling takes about 40s for a typical LSBG.

The deblending procedures are demonstrated in Figure \ref{fig:vanilla_scarlet_demo} with two distinct examples. The one shown in the top panels is a blue LSBG, whereas the one shown on the bottom is a high-$z$ galaxy blended with the outskirts of a nearby galaxy which falls in our initial sample as a false positive. The first column shows the $griz$ color-composite image \citep{Lupton2004}, where the white boxes mark the bounding box of each source in the scene. The gray shades over objects outside of the target bounding box indicate they are masked out during fitting. The second column shows the initial \code{scarlet} models. They look fuzzy because they are constructed from the smoothed image, but they capture the color and the faint outskirts of the LSBG quite well. The optimized model is shown in the middle column of Figure \ref{fig:vanilla_scarlet_demo} after convolving with the PSF, and the fourth column shows the residual image. The fifth column shows the optimized model only for the target object. 

We notice that the optimized model of the blue LSBG captures its color and morphology quite well. For the false-positive case, the optimized model has a notable bright background thanks to the flat-sky component, and the model for the target itself is very red and compact. The reason why it passes our initial selection is that the galaxy outskirts are shredded by \code{SourceExtractor} and happen to have a large size and the right color. But once modeled using \code{scarlet}, it becomes clear that this is just a false positive detection and should be removed. A rubric based on the \code{scarlet} model is therefore needed to help us exclude the false positives. 

% {\bf Here I think belongs a paragraph with a high level summary of the steps and how they work, as above. So a few sentences on peak detection, model initialization, measurement and false positive removal. Then I'd think most of 3.2.1, 3.2.2 could go Appendix, but keeping most of the measurement stuff and visual inspection in the main text.}

\subsubsection{False positive removal}\label{sec:non-par-measurement}

\begin{figure*}
	\vbox{
		\centering
		\includegraphics[width=1.0\linewidth]{deblending_cuts.pdf}
	}
	\caption{The distribution of LSBG candidates from our initial sample in the parameter space as measured from the \code{scarlet} models. We visually inspected 5,000 LSBG candidates and labeled them as \code{candy} (LSBGs with typically dwarf galaxy features), \code{galaxy} (higher surface brightness galaxies or galaxies with spiral galaxy features), and \code{junk} (false positives). We propose a selection based on size, color, surface brightness and morphology (dashed boxes) to effectively remove false positives but still achieves a high completeness. After such cuts (bottom panels), 97\% of false positives and 76\% of non-LSB galaxies are removed according to our visual inspection results. 
	}
	\label{fig:deblending_cuts}
\end{figure*}

After running vanilla \code{scarlet} for LSBG candidates, the target object has been successfully deblended from nearby sources. Unlike other parametric modeling methods, the non-parametric \code{scarlet} model is flexible enough to adequately represent galaxies with complex structures. But the non-parametric nature of the code means that we have to make our own measurements to quantify the size and shape of the \code{scarlet} model. As shown in the last column of Figure \ref{fig:vanilla_scarlet_demo}, we isolate the model of the target object and analyze it using \href{https://statmorph.readthedocs.io/en/latest/}{\code{statmorph}} \citep{statmorph}. The purpose of this analysis is to extract the structural and morphological parameters of the target object. We further use these diagnostics to remove false positives. 

\code{Statmorph} calculates non-parametric morphological and structural parameters, including effective radius, concentration-asymmetry-smoothness statistics \citep[CAS,][]{Conselice2003}, and Gini-M20 \citep{Abraham2003,Lotz2004}. The measurement requires an image, a variance map, and a PSF. For our purpose, the ``image'' is just the \code{scarlet} model convolved with the observed PSF. The variance map and PSF are taken from HSC cutouts. We also force the sky level in \code{Statmorph} to be zero because sky has already been fit when running vanilla \code{scarlet}. 

One of the most important diagnostics is the effective radius of the object. In this paper, we refer to effective radius as the circularized effective radius $r_{e}$, defined as the $r_{e} = r_{\rm eff, sma} \sqrt{b/a}$ where $r_{\rm eff, sma}$ is the effective radius measured along the semi-major axis of the aligned elliptical isophotes, and $b/a$ is the axis ratio of the isophotes. In \code{statmorph}, the effective radius is calculated using elliptical apertures, where the ellipticity $\varepsilon$ is determined by calculating the second moment of the image. The total magnitudes are simply calculated by summing up the model flux in each band, and colors are defined using total magnitudes.\footnote{A Galactic extinction correction is not applied at this point.} The central surface brightness $\mu_0$ is measured by extrapolating the surface brightness profile to $r\to 0$. We define the average surface brightness within the effective radius as 
\begin{equation}\label{eq:mu_eff}
    \mu_{\rm eff} = m - 2.5 \log_{10}(2 \pi r_e^2),
\end{equation}
where $m$ is the total magnitude. 

We also use the Gini-M20 and CAS statistics. The Gini coefficient and M20 statistics \citep{Abraham2003,Lotz2004} quantify how concentrated/extended the flux distribution is across the image. The CAS statistics characterize the concentration $C$, asymmetry $A$, and smoothness $S$ of the light distribution of the object. We refer the reader to \citet{statmorph} for more details on their definitions and implementations. We measure the morphological and structural parameters in $g$-band. %The morphological and structural parameters are the same in different bands by construction in the \code{scarlet} model.

To better guide us on how to use these diagnostics, we visually inspect a subset of LSBG candidates in our initial sample (\S \ref{sec:detection}) and use the classification results to help construct the metrics. To be specific, we randomly selected 5,000 LSBG candidates that are matched with a MW-like host at $0.01 < z < 0.04$ from the NSA catalog (see \S\ref{sec:match} for details). We run vanilla \code{scarlet} for all of them and measure morphological parameters as described above. Then we do visual inspection of the $griz$ color-composite images with 0.5 arcmin on a side. Objects are classified into three types: LSB galaxies with typical dwarf galaxy features (dubbed \code{candy}), high surface brightness galaxies, or galaxies with spiral galaxy features (\code{galaxy}), and false positives including tidal streams, galaxy outskirts blends, and other artifacts (\code{junk}). The difference between \code{candy} and \code{galaxy} can be quite ambiguous. 

The visual inspection was done by coauthors JEG, JG, SH, RB, KC, AG, and EKF. Each object has been inspected by at least one person. In the end, we combine the votes from different people. An object is classified as \code{junk} if the number of votes as \code{junk} is larger than the number of votes as \code{candy} and \code{galaxy}. For objects that are not \code{junk}, if the number of votes as \code{candy} is greater than the votes as \code{galaxy}, we classify this object as \code{candy}. Everything else is classified as \code{galaxy}. In total, there are 1153 \code{candies}, 2150 \code{galaxies}, and 1697 \code{junk}. 

We use these visual inspection results to inform how to use the morphological and structural diagnostics. The distributions of \code{candy} (blue), \code{galaxy} (green), and \code{junk} (red) in the parameter space are shown in the top panels of Figure \ref{fig:deblending_cuts}. Although \code{junk} is scattered throughout the parameter space, a large fraction of the junk comprises small things with very red colors, very low surface brightness, large concentration and large asymmetry. Therefore, we come up with the following selection cuts to remove the false positives:
\begin{itemize}
    \item Color:
    \[0.0 < g-i < 1.2,\quad |(g-r) - 0.7\cdot (g-i)| < 0.25\]
    \item Size: \[1.8 \arcsec < r_e < 12 \arcsec\]
    \item Surface brightness: \[\mu_0(g) > 22.5,\quad 23.0 < \mu_{\rm eff}(g) < 27.5,\]
    \item Morphology: 
    \begin{gather*}
        \varepsilon < 0.65,\quad \mathrm{Gini} < 0.70,\quad M_{20} < -1.1,\\
        \mathrm{Gini} < -0.136\cdot M_{20} + 0.37,\\
        1.8 < C < 3.5,\quad A < 0.8.
    \end{gather*}
\end{itemize}
These criteria (shown as dashed lines in Figure \ref{fig:deblending_cuts}) effectively help us remove objects that are not likely to be LSBGs of interest. The color cuts here are more restrictive than the one used in the initial detection (Sec \ref{sec:detection}) to further remove junk and background high redshift galaxies. We not only remove objects with small sizes, but also exclude objects with large size, very low surface brightness, and high concentration. This is because the scarlet models of some junk can be very concentrated in the center but also very extended. The slanted demarcation line on the Gini-$M_{20}$ diagram is motivated by the line used in \citet{Lotz2008} to separate merging galaxies from normal galaxies. 

The bottom panels in Figure \ref{fig:deblending_cuts} show the objects after applying the above selections. With these cuts, 97\% of the false positives are removed, but the majority (70\%) of \code{candy} still remains. The \code{galaxy} subset also drops by 76\%. This empirical selection based on the non-parametric measurements on the scarlet models successfully removes most of the false positives and a large fraction of background galaxies in our initial sample. Furthermore, the completeness of real LSBG detection remains high. In Section \ref{sec:comp_meas}, we characterize the completeness of this ``deblending'' step by injecting mock \sersic{} galaxies, and we achieve $\sim80\%$ completeness at $\sbeff = 27.0\ \sbunit$ and $>50\%$ completeness at $\sbeff = 27.5\ \sbunit$ (Fig. \ref{fig:completeness}). We emphasize that the vanilla \code{scarlet} modeling and non-parametric measurements are not designed to recover the intrinsic properties of the galaxies. They are used only as a diagnostic tool to remove false positives. We perform more detailed modeling in Section \ref{sec:modeling} to measure galaxy properties for science. 


Many other works have been using machine learning (ML) algorithms to classify LSBG candidates and exclude spurious detection. Among others, \citet{Tanoglidis2021} take the output catalog from \code{SourceExtractor} and use the Support Vector Machine algorithm to classify objects in their initial sample and reduce the number of objects that are visually inspected. \citet{Zaritsky2019,Zaritsky2021,Zaritsky2022} use convolutional neural networks to classify LSBGs into binary classes based on their images. These endeavors certainly help reduce the human labor on visually inspecting tens of thousands of objects. However, training such a supervised ML algorithm relies on the labeled data which should also sample the parameter space fairly evenly. Once labels are assigned (e.g., LSBG v.s. non-LSBG), there will be little way to change later. On the contrary, our cuts based on the measured properties can be tweaked to have higher completeness  (including more junk) or have higher purity (aggressive cuts) depending on the specific science goal. The visual inspection labels only guide us to design those metrics. This measurement-based method is more intuitive, reproducible, and transferable. We will also explore machine learning methods and compare with our deblending cuts in future works. Future work will also utilize information about Milky Way dust to exclude spurious detection on Galactic cirrus \citep[e.g.,][]{Zaritsky2021,Zaritsky2022}.

% Tanoglidis doesn't estimate measurement uncertainty. Doesn't systematically estimate completeness, only validates the comp using deeper fornax observations. Also, the artifact of HSC is less compared with DES/DECaLS?
% SMUDGES: Their sample might be cleaner thanks to their wavelet detection emphasizing specific spatial frequency. But they still need a lot of visual inspection.

\subsection{Modeling}\label{sec:modeling}

% Similar to many other modeling codes (e.g., \code{galfit} \citealt{galfit}, \code{imfit} \citealt{imfit}), \code{scarlet} is initialized with the positions of detected peaks. 

Although the vanilla \code{scarlet} modeling provides useful information to help us remove false positives, it does not necessarily give us reliable estimations on size, color, total magnitude, etc. In fact, the vanilla \code{scarlet} model systematically under-estimates the size and the total flux of LSB objects. This is because the faint outskirts of LSBGs are typically not modeled well due to the non-parametric nature of vanilla \code{scarlet}. When doing parametric modeling such as \sersic{} fitting, we effectively apply radial averaging within elliptical annuli and boost the signal-to-noise ratio by binning pixels. In other words, pixels within the same isophotal annulus are assumed to have the same intensity thus are strongly correlated. However, the non-parametric nature of vanilla \code{scarlet} only assumes monotonicity, which imposes quite weak correlations among pixels. Consequently, it is hard for the non-parametric modeling to probe very low surface brightness features. Furthermore, the monotonicity constraint stops the model from growing in certain directions if there is another source along this direction, otherwise monotonicity is broken.\footnote{This issue can be seen in the third column of Figure \ref{fig:vanilla_scarlet_demo} where the model of the target object never exceeds the other objects next to it.} As a result, the non-parametric model often does not capture well the LSB outskirts of LSBG, thus biasing the measurements. In the following, we explore a novel method to perform robust parametric modeling and measurement while exploiting the deblending power of \code{scarlet}.

A traditional way of doing parametric modeling is to mask out contaminants based on the detection segmentation map and fit a model to the masked image. However, such fitting results are very sensitive to the masking scheme \citep[e.g.,][]{Greco2018} and sky background. A possible solution to this problem is to simultaneously model all the objects in the cutout and the sky using parametric models (e.g., in DECaLS, \citealt{Lang2016,Dey2019}). In this work, we combine the advantage of parametric modeling with the power of deblending in \code{scarlet}. To be specific, we follow the spirit of deblending as described in \S\ref{sec:deblending}, but replace the non-parametric model for the target galaxy with a parametric model. In this way, the LSB outskirts of LSBGs can be better captured with the parametric model and the impact of contaminants is minimized because they are modeled simultaneously in all bands with non-parametric models. The parametric model for the target object can also extend to the whole scene and not stopped by neighboring objects. 

In this work, we use the Spergel surface brightness profile \citep{Spergel2010} to model the LSBGs (see Appendix \ref{ap:spergel} for details). The Spergel profile is motivated by having a simple analytical expression in Fourier space, making it easy to convolve with a PSF. Similar to the \sersic{} index, the parameter $\nu$ in \eqref{eq:spergel} (dubbed as the ``Spergel index'') controls the concentration of the light profile. As shown in Appendix \ref{ap:spergel}, the Spergel profile approximates the \sersic{} profile very well over the range of \sersic{} index that is relevant to the study of LSBGs.

We initialize the Spergel model for the target object differently, while other objects are still initialized in the same way as in the deblending step (Section \ref{sec:deblending}). First, we initialize a vanilla \code{scarlet} model for the target object, and we measure the effective radius $r_e$, total flux, and the shape of the scarlet model. We use these numbers to initialize the Spergel profile. The size of the bounding box is also updated to be the maximum between 250 pixels and $10\, r_e$. Peaks within this bounding box (other than the target) are modeled using vanilla \code{scarlet}. For the target, we still require a positivity constraint, and the monotonicity is automatically satisfied by the Spergel profile. After optimization, we take the $r_0$ in Equation \eqref{eq:spergel} as the circularized half-light radius $r_e$, and take $L_0$ as the total flux. The average surface brightness $\overline{\mu}_{\rm eff}$ is calculated in the same way as in \eqref{eq:mu_eff}. The Spergel modeling results are used for studying the properties of LSBGs. 
In the following section, we assess the quality of the Spergel modeling by injecting mock \sersic{} galaxies and comparing the recovered properties with truth. 

% \textbf{
% Based on the modeling results, we apply the same color cuts as in Section \ref{sec:deblending} and a size cut of $1.6\arcsec < r_e < 15\arcsec$ to further remove unreliable fits and false positives. }
% \jiaxuan{Check if this is considered in the completeness.}

\subsection{Completeness and Measurement Uncertainty}\label{sec:comp_meas}
\begin{figure*}
	\vbox{ 
		%\vskip -10mm
		\centering
		\includegraphics[width=1\linewidth]{completeness.pdf}
	}
	\caption{The completeness of our LSBG search as a function of effective radius $r_e$ and the average surface brightness $\sbeff$. The overall completeness (right panel) comprises the detection completeness (left panel, see \S \ref{sec:detection}) and deblending completeness (middle panel, see \S \ref{sec:deblending}) where the false positives are removed. The dashed lines in the right panel highlight the 70\%, 50\%, and 20\% completeness contours. The completeness drops as we go to fainter surface brightness and smaller size. We are $>70\%$ complete to $\sbeff < 26.5\ \sbunit$ and $3\arcsec < r_e < 14\arcsec$, and $>50\%$ complete to $\sbeff \leqslant 27.0\ \sbunit$. 
	}
	\label{fig:completeness}
\end{figure*}

The completeness of the LSBG search is important not only for understanding our search efficiency and improving the method in the future, but also for deriving the completeness-corrected results for LSBG sciences. Similar to many other works \citep[e.g.,][]{vdBurg2017,Zaritsky2021,CarlstenELVES2022,Greene2022}, we derive the completeness by simulating mock LSBGs and recovering them from the image. Using these mock galaxies, we also characterize the measurement bias and uncertainty by comparing the measured properties with the truth. We present the main steps below and describe details in Appendix \ref{ap:comp_meas_unc}.

A real LSBG can be missed in the detection step (\S \ref{sec:detection}) or be excluded in the deblending step (\S \ref{sec:deblending}). Thus, the overall completeness is a combination of detection completeness and deblending completeness. We performed a large suite of image simulations to derive the completeness. We inject single \sersic{} light profile \citep{Sersic1963} to the coadd images and try do recover them using the detection method (\S\ref{sec:detection}), model them using vanilla \code{scarlet} (\S\ref{sec:deblending}), and apply the deblending cuts. The mock \sersic{} galaxies span a wide range in size ($2\arcsec \leqslant r_{e} \leqslant 21\arcsec$) and surface brightness ($23 \leqslant \overline{\mu}_{\rm eff}(g) \leqslant 28.5\ \mathrm{mag\ arcsec^{-2}}$). The detection completeness is defined as the number of detected objects divided by the number of injected objects, whereas the deblending completeness is defined as the fraction of objects remaining after the deblending cuts. 

We find that the completeness mainly depends on the size and surface brightness. As shown in Figure \ref{fig:completeness}, the detection completeness remains high across different sizes. It drops below 50\% when the surface brightness gets fainter than $\sbeff = 27.5\ \sbunit$. The deblending completeness is very high at the bright end, but starts to decline with increasing size and decreasing surface brightness. Mock galaxies fainter than $\sbeff > 27.5\ \sbunit$ are mostly removed by the deblending step, likely due to the blending of other small compact sources with the mock galaxy. Interestingly, the detection completeness and deblending completeness all drop below 40\% around $\sbeff=27.5\ \sbunit$. In this sense, the detection and deblending cooperate well in terms of completeness. 

The combined completeness is shown in the right panel of Figure \ref{fig:completeness}. The dashed lines highlight the 70\%, 50\%, and 20\% completeness contours. The completeness drops as we go to fainter surface brightness and smaller size. We are $>70\%$ complete to $\sbeff < 26.5\ \sbunit$ and $3\arcsec < r_e < 14\arcsec$, and $>50\%$ complete to $\sbeff \leqslant 27.0\ \sbunit$. Although \sersic{} galaxies do not necessarily resemble the morphology of real LSBGs, our completeness based on \sersic{} model still sets a baseline or upper limit on the real completeness. In the future, we might be more capable of generating realistic LSBGs using deep learning techniques (e.g., diffusion model). 

We compare our completeness with that of other LSBG searches. Our completeness is most similar to \citetalias{Greco2018}, where they also reach $\sim 50\%$ at $\sbeff \approx 27.0\ \sbunit$ and $3\arcsec < r_e < 12\arcsec$ \citep{Kado-Fong2021,Greene2022}, but their completeness decreases more quickly with increasing size. \citet{CarlstenELVES2022} use a similar search algorithm to \citetalias{Greco2018} and ours and reach a completeness of $\sim 50\%$ at $\mu_0(g)\approx 26.5\ \sbunit$ or equivalently $\sbeff = 27.5\ \sbunit$ assuming a Sersic index $n \approx 1$. \citet{vdBurg2017} searched LSBGs from the ESO Kilo-Degree Survey data and achieved $\sim 50\%$ completeness at $\sbeffr\approx 26.0\ \sbunit$ or equivalently $\sbeff \approx 26.6\ \sbunit$. \citet{Zaritsky2021} focus on searching for LSBGs in the SDSS Stripe82 region using the Dark Energy Camera Legacy Survey (DECaLS, \citealt{Dey2019}) images and reach $\sim 25\%$ completeness at $\mu_{0}(g) \approx 25.5\ \sbunit$ or equivalently $\sbeff \approx 26.5\ \sbunit$ for $n=1$. Their low completeness might be explained by the fact that they remove a large fraction of the sky contaminated by Galactic dust. \citet{Tanoglidis2021} conducted an LSBG search using the Dark Energy Survey (DES, \citealt{Abbot2018}) data and conclude an overall completeness of $\sim 40\%$ by comparing their catalog with the deeper observations in the Fornax cluster. \citet{Kado-Fong2021} estimate the completeness of \citet{Tanoglidis2021} by comparing with \citetalias{Greco2018} and find a completeness similar to \citetalias{Greco2018} at $\sbeff = 25.5\ \sbunit$ but much smaller completeness above $\sbeff > 26.5\ \sbunit$. The Dragonfly Wide Field Survey \citep{Danieli2020} reaches $\sim 29\ \sbunit$ for $>3\sigma$ detection on scales $\sim 1\arcmin$, which is deeper than all other surveys by sacrificing spatial resolution. In summary, our search achieves an overall high completeness compared with other works, and we demonstrate the great potential of HSC data on LSBG studies \citep[e.g.,][]{Huang2018,Kado-Fong2018}. 

\begin{figure*}
	\vbox{ 
		\centering
		\includegraphics[width=1\linewidth]{meas_error_spergel.pdf}
	}
    \caption{The measurement bias and uncertainty as a function of measured angular size $r_e$ and surface brightness $\sbeff$. The colors show the bias term while the dashed contours show the constant uncertainty lines. We derive the bias in the total magnitudes from the combined bias in $r_e$, $\sbeff$, and color. We find that the bias and uncertainty in our measurement increase toward the fainter end. We apply the bias correction to the real LSBGs before studying their physical properties.}
    \label{fig:meas_err}
\end{figure*}

\vspace{1em}

The size, magnitude, surface brightness, and shape of LSBGs are hard to characterize because of the low surface brightness nature. We characterize the measurement quality by comparing the Spergel modeling results with the ground truth for mock \sersic{} galaxies. We find that the measured effective radius $r_e$ is biased to be smaller than the truth, and the bias depends on the surface brightness and the angular size. For surface brightnesses fainter than $27\ \sbunit$, the measured size can be much smaller than the truth. As a result, the total magnitude $m$ and the average surface brightness $\overline{\mu}_{\rm eff}$ are also measured to be fainter than reality. Similar to \citet{Zaritsky2021}, we derive the measurement biases and uncertainties in size, surface brightness, total magnitude, and color as a function of size and surface brightness. We refer interested readers to Appendix \ref{ap:comp_meas_unc} for details. The main results are shown in Figure \ref{fig:meas_err}, where the colors show the bias terms as a function of the measured size and surface brightness. The size and surface brightness bias grows with increasing surface brightness. The bias term is not a monotonic function of size because the size here is not intrinsic size. It is the bias in size measurement that makes galaxies with large intrinsic $r_e$ pile up around $r_e\sim 6\arcsec$. We also find the bias in $g-i$ color to be quite small. The measurement errors $\sigma(X)$ are shown in Figure \ref{fig:meas_err} as contours, and they have the same units as the biases. We set minimum uncertainties to be $\sigma(r_e) \geqslant 0.3,\ \sigma(\overline{\mu}_{\rm eff}) \geqslant 0.05,\ \sigma(g-i) \geqslant 0.05$ to avoid meaningless uncertainty due to small statistics. 

In the following sections, we apply the bias corrections to the real LSBGs, estimate the measurement uncertainties, and evaluate the completeness based on bias-corrected properties. We also incorporate the measurement errors into the science figures. The values presented in our catalogs (Table \ref{tab:catalog}) are already bias-corrected, but we do provide the bias values for reference. 

% {\bf I think you can combine 3.4.1 and 3.4.2, and put most of this into the Appendix, just introduce the idea of the simulations and show the figure. I'd keep the comparison with other works in the main body of the paper.}


\section{Mass-size outliers in Milky-Way analogs}\label{sec:sample_construction}
The goal of this paper is to study the mass-size outliers in the satellite systems around MW-like galaxies. However, distance information is needed to convert the observed size and magnitude to the physical size and stellar mass. It is well-known that getting the distances to LSBGs is one of the major obstacles for studying their properties. Beside direct distance measurements (e.g., SAGA, ELVES, \citealt{Kadowaki2021}), it is common to assume a distance to a LSBG by associating it with a host galaxy based on the projected angular distance \citep[e.g.,][]{vanDokkum2015,vdBurg2016,Wang2021,Zaritsky2022,Nashimoto2022}. A statistical background subtraction is then needed to account for the contribution from background and foreground galaxies. Furthermore, cross-correlation between an LSBG sample and a host galaxy sample can also reveal the distance distribution of LSBGs \citep{Greene2022}. These methods are also complemented by recent machine learning techniques \citep{Baxter2021,xSAGA-I}. In this Section, we cross-match our LSBG catalog with MW-like galaxies in the NSA catalog. For the LSBGs matched with MW analogs, we run the deblending step to exclude false positives (\S\ref{sec:deblending}) and then run Spergel modeling to measure the properties of LSBGs (\S\ref{sec:modeling}). In the end, we construct the samples of mass-size outliers in the satellites of MW analogs. 

\subsection{Matching with Milky-Way analogs}\label{sec:match}
The properties of the Milky Way itself vary in the literature \citep{Licquia2015,Bland-Hawthorn2016} and the definitions of MW analogs are different among different studies. In the SAGA survey \citep{SAGA-I,SAGA-II}, MW analogs are selected based on their absolute $K$-band magnitude $-23 > M_K > -24.6$~mag, which is derived using abundance matching by assuming a simple galaxy-halo connection model (see Fig. 2 in \citealt{SAGA-I}). This luminosity range approximately corresponds to a stellar mass range of $10.2 < \log\, M_\star/M_\odot < 11.0$. SAGA also requires that the MW analogs be in isolation (without nearby bright galaxies) and lie in a redshift range of $0.005 < z < 0.01$ (20--40 Mpc). In the ELVES survey \citep{ELVES-I,ELVES-II,CarlstenELVES2022}, the requirements for MW-like host is loosened to be $M_K < -22.1$~mag ($M_\star > 10^{9.9}\ M_\odot$) because the probed volume by ELVES ($D<12$ Mpc) is smaller than that of SAGA. In this work, we choose the stellar mass range of MW analogs to be $10.2 < \log\, M_\star/M_\odot < 11.2$, which is simply a 1 dex bin centered at the measured stellar mass of the Milky Way ($M_{\star, \mathrm{MW}}\approx 10^{10.7}\ M_\odot$, \citealt{Licquia2015}). MW analogs selected using this definition are very close to those in SAGA but are slightly less massive than the ELVES hosts since ELVES has several groups more massive than SAGA hosts.

Mass-size outliers, by definition, are relatively scarce compared with normal-sized satellites in MW analogs \citep{SAGA-II,CarlstenELVES2022}. It is therefore helpful to probe a larger volume to obtain good statistics. We choose our redshift range to be $0.01 < z < 0.04$, which makes sure that we can detect a significant number of faint dwarf galaxies around MW-like hosts. The depth of HSC limits our detection below $z \approx 0.04$ since dwarf galaxies will be too small and too faint to be detected beyond that distance\footnote{Our search only includes objects larger than $r_e\approx 2\arcsec$, which corresponds to $M_\star \sim 10^{8.5}\ M_\odot$ at $z=0.04$ assuming the \textit{average} mass-size relation from \citet{ELVES-I}. Going to higher redshift will make it more difficult to detect low-mass dwarfs.}. We also exclude galaxies $z<0.01$ because (1) the number of UDGs will be very small due to the small volume; (2) their large angular size makes them more vulnerable to be shredded in the detection, so including them will introduce a number of spurious LSB objects. We do not discriminate our hosts to be isolated or paired.
%Our LSBG sample complements the ELVES sample and SAGA sample in our redshift range. 

After applying the stellar mass and redshift cuts to the NSA catalog (\S\ref{sec:data}), there are 23,218 MW-like galaxies. We match them to the initial LSBG sample (described in \S \ref{sec:detection}, before the deblending step) as follows. For a given MW-like galaxy, we first calculate its virial radius $R_{\rm vir}$ assuming the stellar-to-halo mass relation in \citet{Behroozi2010}. It turns out that 40\% of our hosts have virial radii larger than 300 kpc, which is commonly used for the virial radius of our Milky Way. Then we identify any LSBG that falls into the projected virial radius of the host on the sky. If one LSBG is matched to multiple hosts, we assign it to the nearest host based on the separation normalized by host virial radius. Finally, we have 901 MW-like hosts and 10,579 LSBG candidates associated with them. 

However, there are still a significant fraction of spurious objects in this cross-matched LSBG sample, including galactic cirrus, tidal streams, shredded large galaxies, and compact sources blended with galaxy outskirts. As described in \S\ref{sec:deblending}, we perform a deblending step to effectively remove these spurious objects. There are 2,673 objects left after the deblending cuts. Next, we model them using the Spergel profile as described in \S\ref{sec:modeling} and obtain their photometry and structural parameters from the best-fit models. 
% Based on the modeling results, we apply the same color cuts as in Section \ref{sec:deblending} and a size cut of $1.6\arcsec < r_e < 15\arcsec$ to further remove unreliable fits and false positives. 
We also remove duplicated objects. In the end, we have 2,510 LSBG candidates as our final LSBG sample. We note that these LSBGs are matched to MW hosts in projection but do not have direct distance measurements. 

We apply the measurement bias correction (\S \ref{sec:comp_meas}) and assign completeness after bias correction. In the end, we correct for the effect of Galactic extinction on colors based on \citet{SFD1998,Schlafly2011}. We derive the stellar masses of cross-matched LSBG from the Spergel model fitting and a relation between color and mass-to-light ratio $M_{\star}/L$ from \citet{Into2013}:
\begin{align*}
&\log \left(M_{\star} / L_{g}\right)=1.774\,(g-r)-0.783, \\
&\log \left(M_{\star} / L_{g}\right)=1.297\,(g-i)-0.855.
\end{align*}
Because we have both $g-r$ and $g-i$ color available from the model fitting, we use the average of the $M_{\star}/L$ derived from the two colors for calculating stellar mass. We assume the solar absolute magnitude in $g$-band to be 5.03 \citep{Willmer2018}. 

\subsection{Mass-size outlier sample}\label{sec:sample}
The concept of UDG was proposed to describe dwarf galaxies with unusually large size ($r_e>1.5$ kpc) and diffuse light distribution ($\sbcen > 24.0\ \sbunit$). Although UDGs are large in size, the constant size cut does not account for the fact that the galaxy size is strongly correlated with its mass, known as the mass-size relation \citep[e.g.,][]{Graham2003,Trujillo2007,vanDokkum2013,Cappellari2013,Lange2015}. 

A more intuitive definition for diffuse and puffy dwarf galaxies (i.e., mass-size outliers) should be based on a mass-size relation and its scatter. Following this thread, we propose a new definition for mass-size outliers dubbed as \textit{ultra-puffy galaxies} (UPGs). In this section, we describe the definitions of UPG and contrast it with UDG, then we select UDGs and UPGs from our LSBGs matched with MW hosts. 
% Ideally, galaxies that are $1.5\sigma$ or $2\sigma$ above the average mass-size relation represent the large-size tail of the population in a homogeneous manner. 

\subsubsection{UPGs}
To select mass-size outliers, a mass-size relation and its scatter are needed. However in general, the slope of the mass-size relation in the nearby Universe has been shown to be color- and morphology-dependent for galaxies above $M_\star > 10^{9}\ M_\odot$: blue star-forming galaxies have a shallower slope than the red quenched galaxies \citep[e.g.,][]{Lange2015}. Fortunately, in the dwarf galaxy regime ($10^{5.5} < M_\star < 10^{8.5}$), \citet{ELVES-I} derive the mass-size relation from the satellites of MW analogs in the Local Volume and show that the average mass-size relation is quite universal: the slope and intercept are not sensitive to the color or morphology of dwarf galaxies. The residual from the mass-size relation follows a log-normal distribution reasonably well. This gives us a chance to define a subset of dwarf galaxies that are size-outliers with respect to the average mass-size relation. It is worth noting that a definition based on the mass-size relation was already used in simulation studies where simulations cannot reproduce the observed mass-size relation \citep[e.g.,][]{Benavides2021,Benavides2022}. 

Taking the measured average mass-size relation from \citet{ELVES-I} $\log\, (r_e/\mathrm{pc}) = 0.25\, \log\, (M_\star/M_\odot) + 1.07$ and a scatter of $\sigma=0.18$ dex, we define a population of ultra-puffy galaxies to be galaxies that are $>1.5\sigma$ above the average mass-size relation. We choose $1.5\sigma$ just because it gives us a statistically significant sample. One can certainly select $>2\sigma$ or $>3\sigma$ UPGs if a larger LSBG sample is available. The UPGs are not absolutely large in size or absolutely faint in surface brightness, but they are outliers in size for their stellar masses. The definition of UPG based on the observed mass-size relation is more physically-motivated, and better allows us to study size outliers at a given stellar mass. We further discuss the advantages and disadvantages of UDG and UPG in \S\ref{sec:mass-size} and \S\ref{sec:discussion}.

There are 362 objects in our LSBG candidates that fall $1.5\sigma$ above the ridge line of the mass-size relation.\footnote{We note that the mass-size relation in \citet{ELVES-I} is derived for a mass range of $10^{5.5}\ M_\odot < M_\star < 10^{8.5}\ M_\odot$. We extrapolate the mass-size relation to $M_\star \sim 10^9\ M_\odot$ to define our UPG sample.} After removing spurious objects after visual inspection and removing objects with completeness less than 0.1, we have 337 galaxies in our UPG sample. These UPGs are associated with 239 hosts. The total sky area occupied by UPG hosts (out to 1 $R_{\rm{vir}}$) is 32.37 deg$^{2}$. The catalogs are available in machine-readable format online\footnote{\url{https://github.com/AstroJacobLi/kuaizi/blob/master/data/}}. We demonstrate the catalog format in Table \ref{tab:catalog}. 

%%%%%%%%%%%%%%%%%%%%%%%%%%%%%%%%
\subsubsection{UDGs}
Albeit the advantage of UPGs, there are very little literature on UPGs for us to compare with. Thus, we also select a UDG sample from our LSBGs in order to compare with existing observations and simulations in different environments. The UDG sample is usually defined based on the surface brightness and the physical size of the galaxy. However, there are many different criteria in literature: \citet{vanDokkum2015} define UDGs to have effective radii $r_e$ larger than 1.5 kpc and central surface brightness $\mu_0(g)$ fainter than $24.0\ \sbunit$; other groups also use the surface brightness at $r_e$ \citep[e.g.,][]{DiCintio2017,Cardona-Barrero2020} or the average surface brightness within $r_e$ to define UDGs \citep[e.g.,][]{Koda2015,Yagi2016,vdBurg2016,Leisman2017,Martin2019}. The size criterion also varies from $r_e > 1$ kpc to $r_e > 1.5$ kpc. \citet{vanNest2022} explore these definitions in simulations and find that different definitions of ``UDG'' can drastically change the selected subset of dwarfs, therefore affecting our understanding of the UDG population.

In practice, the measured central surface brightness might be biased by nuclear star clusters \citep{Neumayer2020,ELVES-II,Somalwar2020} or other contaminants. Therefore in this work, we use the average surface brightness within the effective radius $\sbeff$ to define UDGs. The difference between the average surface brightness and the central surface brightness can be analytically calculated for a \sersic{} profile: $\overline{\mu}_{\mathrm{eff}} - \mu_0 = 1.124$ for $n=1$ and 0.796 for $n=0.8$ \citep{Graham2005,Yagi2016}. Since dwarf galaxies and UDGs typically have \sersic{} index of $0.8 < n < 1.2$ \citep[e.g.,][]{vanDokkum2015,ELVES-I}, we take $\overline{\mu}_{\mathrm{eff}} - \mu_0 = 1.0$ as an average value to convert $\mu_0$ to $\mu_{\mathrm{eff}}$. In this work, we define UDGs to be galaxies with $r_e+\sigma(r_e) > 1.5$ kpc and $\sbeff + \sigma(\sbeff) > 25\ \sbunit$, where we also take the $1\sigma$ measurement errors into account. As a result, there are a few objects with smaller size but large uncertainty scattered in our UDG sample. This definition maximizes the consistency with the definition in \citet{vanDokkum2015} while not losing UDGs harboring nuclear star clusters.

Among the 2,510 LSBG candidates, there are 432 objects satisfying the UDG definition. We did a final visual inspection for these objects and excluded 16 objects that are false positives including blends and Galactic cirrus. We also removed another 4 objects having completeness less than 0.1. In the end, we obtain our UDG sample with 412 objects associated with 258 hosts. The total sky area occupied by UDG hosts (out to 1 $R_{\rm{vir}}$) is 32.71 deg$^{2}$. We describe the properties of the mass-size outliers (including UPGs and UDGs) in \S\ref{sec:mass-size}.

\subsection{Background contamination fraction}\label{sec:bkg}
% The major weakness of LSBG searches in imaging data is that distance information is typically unknown. 
Although we have matched LSBGs to MW analogs by proximity, the physical affiliation of LSBGs to the host is not guaranteed. As a result, a certain fraction of galaxies in our mass-size outlier samples might just be foreground or background galaxies that fall within the virial radius of the host in projection. Given the volume probed by our search, contamination is most likely to be dominated by background galaxies. In the section, we try estimate the contamination fraction empirically by studying the number of those LSBGs that mimic UPGs and UDGs by randomly matching LSBGs to our MW hosts.

We randomly select a continuous patch of sky of 24 deg$^{2}$ in HSC PDR2 regardless of whether containing MW analogs\footnote{$345\ \deg < \rm{R.A.} < 351\ \deg$, $-1.5\ \deg < \rm{Dec} < 2.5\ \deg$, which is not dominated by Galactic cirrus and stars.}. Then we perform the same deblending and modeling steps for the 2,707 LSBGs detected in this region. We also removed objects that are already in our mass-size outlier samples and objects with completeness less than 0.1. Furthermore, we also did the same visual inspection as for UDGs and UPGs to remove false positives. In the end, there are 480 LSBGs (excluding false positives) representing a population of possible contaminants for the UDG and UPG samples. Next, because both UDG and UPG are defined based on a physical size, we randomly associate the 480 LSBGs to the hosts of our UDGs and UPGs, respectively. Then we calculate a physical size for each LSBG and classify an LSBG as a ``fake`` UDG or UPG if it satisfies the corresponding definition. We repeat such random matching 200 times. In the end, we obtain 11,458 fake UDGs and 11,867 fake UPGs. These UDGs and UPGs are ``fake'' only in the sense of being associated with random MW hosts. 

In this way, the number density of fake UDGs is estimated to be $S_{\rm UDG} = 1.94\pm0.03\ \mathrm{deg}^{-2}$, and the number density of fake UPGs is $S_{\rm UPG} = 2.12\pm0.03\ \mathrm{deg}^{-2}$. The catalogs of fake UDG and UPGs are also available online\footnote{\url{https://github.com/AstroJacobLi/kuaizi/tree/master/data}}. Using the number densities and the probed area of our survey, we calculate the contamination fraction for both the UDG and UPG samples. For the UDG sample, we find $f_{\rm contam}^{\rm UDG} \approx 16\%$; for the UPG sample, the contamination fraction is $f_{\rm contam}^{\rm UPG} \approx 20\%$. We take the contamination fraction into account when calculating the abundances (\S\ref{sec:results}).

% One of the main goals of this paper is to study the star-forming properties of mass-size outliers in satellites of MW analogs. The star formation rate of galaxies can be indicated by their broad-band colors, and we use color to calculate the quenched fraction of UDGs and UPGs (see \S \ref{sec:quench} for more details). However, background contaminants affect the color distribution of the UDG (UPG) sample, and thus affect the calculated quenched fraction. If the fake UDGs (UPGs) have different properties from the real ones, we can use this information to derive a probability that an object is a real UDG (UPG) at the proposed distance. 
% If the fake UDGs (UPGs) span a different region from the true UDGs in the observable space, we can use these observables to assign weights to each object in our UDG sample indicating the possibility of being a true UDG. 
% Indeed, we find that the $g-i$ color distribution of the fake UDG (UPG) sample is bluer than that of the true UDG (UPG) sample, which motivates us to assign importance weights based on the $g-i$ color. Taking UDGs as an example, we first compute the normalized histograms of $g-i$ colors (denoted as $\lambda_k$ at the $(g-i)_{k}$ bin) for both true UDG and fake UDG samples. Then we multiply $f_{\rm contam}^{\rm UDG}$ with the histogram of fake UDG sample ($\lambda_k^{\rm fake}$) such that the histogram sums up to the average contamination fraction. The weight assigned to UDGs in color bin $(g-i)_k$ is estimated to be $w_k = \max\,(1 - f_{\rm contam} \lambda_k^{\rm fake} / \lambda_k^{\rm true}, 0)$. This weight stands for a possibility of not being a contaminant. The weights of UPGs are assigned following the same procedure. We find this color-based contamination subtraction scheme is sufficient for this work since the contamination fraction is small. %Our main results shown in Section \ref{sec:quench} are robust against contamination subtraction. 


\section{Results}\label{sec:results}
In this section, we present a statistical analysis for mass-size outliers in the satellites of MW analogs at $0.01 < z < 0.04$. We first compare the distributions of UDGs and UPGs on the size-surface brightness plane and mass-size plane to gain intuition on the differences between UDG and UPG definitions. Then we present the abundances of UDGs and UPGs and compare with literature. 

\subsection{Properties of mass-size outliers}\label{sec:mass-size}
\begin{figure*}
	\vbox{ 
		\centering
		\includegraphics[width=1\linewidth]{udg_upg_sample.pdf}
	}
    \caption{The distribution ultra-diffuse galaxies (UDGs, \textit{left}) and ultra-puffy galaxies (UPGs, \textit{right}) on the size-surface brightness plane. The UPGs is defined to be galaxies that are $1.5\sigma$ above the average mass-size relation in \citet{ELVES-I}. The samples are split into two color bins and shown in red ($g-i>0.8$) and blue ($g-i<0.8$). The marginal histograms are not normalized to highlight the relative number of red and blue galaxies. Compared with the UDG sample, the UPG sample includes blue galaxies with surface brightness higher than the UDG cut ($\sbeff < 25\ \sbunit$) and excludes red galaxies at $25 < \sbeff < 26\ \sbunit$, due to the color-$M_*/L$ dependence.
    }
    \label{fig:udg_upg_re_mu}
\end{figure*}

\begin{figure*}
	\vbox{ 
		\centering
		\includegraphics[width=1\linewidth]{mass_size_plane_new.pdf}
	}
    \caption{\textit{Left panel}: a schematic diagram showing the UDG size cut (gray band), survey limit (blue shade), the average mass-size relation (dark green dashed line) and $1.5\sigma$ above it (light green line). The solid lines are constant surface brightness cuts at $\sbeff=25\ \sbunit$ for two different colors $g-i=0.4$ (blue) and $g-i=0.8$ (red). The constant surface stellar mass density line is in purple, which is parallel to the solid lines. \textit{Right panel}: The distributions of UDGs and UPGs on the mass-size plane. UDGs that are also classified as UPGs are shown as orange squares, while UDGs that do not satisfy the UPG definition are marked as red stars. Blue triangles are UPGs that are not classified as UDGs. UDG sample includes a significant amount of galaxies that are not mass-size outliers (below the $1.5\sigma$ line) which are red in color because the surface brightness cuts are different on mass-size plane for blue and red galaxies.
    }
    \label{fig:mass_size}
\end{figure*}

In order to build a more comprehensive understanding on our UDG and UPG sample, we show their distributions on the size-surface brightness plane in Figure \ref{fig:udg_upg_re_mu}. The galaxies are split into two color bins and are shown in blue ($g-i < 0.8$) and red ($g-i > 0.8$), respectively. We note that we do not apply any background contamination correction in Figure \ref{fig:udg_upg_re_mu} because such correction can only be done in a statistical sense. The error bars correspond to $1\sigma$ measurement uncertainties (\S\ref{sec:comp_meas}). Some galaxies with size smaller than 1.5 kpc are still scattered in the UDG sample because their sizes have large uncertainty. The two marginal plots show the unnormalized histograms of galaxies in the two color bins. The numbers of red and blue galaxies are similar in the UDG sample, but there are more blue galaxies than red ones in the UPG sample. Since there is no hard surface brightness cut for UPGs, the UPG sample includes many blue galaxies with brighter surface brightness ($23 \lesssim \sbeff < 25\ \sbunit$). Due to the surface brightness cut ($\sbeff > 23\ \sbunit$) in the deblending step, there is little chance for a LSB disk galaxy with bright bulge to scatter in the UPG sample. The UPG sample also loses a significant fraction of red galaxies at $25 < \sbeff < 26\ \sbunit$ because their stellar masses are too high for their sizes to make them being size-outliers. 


We show UDGs and UPGs on the mass-size plane in Figure \ref{fig:mass_size}. To orient the reader to the mass-size plane, we first show a schematic diagram in the left panel where the lines and regions relevant to the definition of UDG and UPG are highlighted. The light blue shade shows the survey limit where we cannot effectively probe LSBGs fainter than $\sbeff \gtrsim 27.5\ \sbunit$ (see \S\ref{sec:comp_meas}). 
For defining UDGs, a size cut and a surface brightness cut are needed. We show the $r_e = 1.5$ kpc cut as a gray band to acknowledge the fact that smaller galaxies with large size uncertainty can also get into the UDG sample. The slanted solid lines show the constant surface brightness lines with $\sbeff = 25.0\ \sbunit$ for two different colors ($g-i=0.4$ and $g-i=0.8$). For our redshift range, the cosmological dimming effect is negligible and the surface brightness is a constant with distance. The constant surface brightness lines depends on color because of the color-$M_\star/L$ relation. We also show the constant surface stellar mass density ($\mu_\star$) line, which is parallel to the constant surface brightness lines because the surface mass density is proportional to surface brightness multiplied by mass-to-light ratio. Therefore, at a given size, blue UDGs are less massive than red UDGs; at a given stellar mass, blue UDGs are larger than red UDGs in size. Such a constant surface brightness cut leads to a UDG population that is inhomogeneous in surface mass density. 

In the left panel of Figure \ref{fig:mass_size}, the dark green dashed line shows the average mass-size relation from \citep{ELVES-I}, and the lighter green line shows $1.5\sigma$ above the average mass-size relation assuming the scatter of the mass-size relation to be $\sigma=0.18$ dex \citep{ELVES-I}. UPGs would lie above the $1.5\sigma$ line by construction, and there will be overlap between UDGs and UPGs. The mass-size relation is shallower than the constant surface brightness and surface mass density lines.

We highlight the similarity and difference between UDG and UPG in the right panel. UDGs that are also classified as UPGs are shown as orange squares, while UDGs that do not satisfy the UPG definition are marked as red stars. Blue triangles correspond to UPGs that are not classified as UDGs. Limited by the detection limit and also the fact that our hosts are more distant than the Local Volume, our samples only reach to $M_\star \approx 10^{6.8}\ M_\odot$ which is $\sim 1$ dex higher than ELVES. 
Compared with the UPG sample, we find that the UDG sample (orange squares + red stars) comprises a number of galaxies below the $1.5\sigma$ line, but also loses several low-mass galaxies that are above $1.5\sigma$ line but have sizes smaller than 1.5 kpc. For the UDG sample, the stellar mass range only reaches $M_\star\approx 10^{8.3}\ M_\odot$ because galaxies with higher stellar mass are too bright to satisfy the surface brightness criterion for UDG. On the contrary, since the UPG definition does not have a hard surface brightness cut, it includes many galaxies that are more massive than UDGs as long as it has an extraordinarily large size. As a consequence, background spiral galaxies could dominate the UPG sample at the high-mass end. 

\jiaxuan{Discuss whether UPGs are LSB disks?}


% discriminative for blue and red galaxies. 

% As we will introduce in Section \ref{sec:quench}, we use a color cut $(g-i)_{Q}$ to separate galaxies with active star formation from quiescent ones. Here we define $\Delta(\mathrm{Quench}) = (g-i) - (g-i)_{Q}$ as an indicator for the star formation rate of the galaxy. Higher $\Delta(\mathrm{Quench})$ indicates that the galaxy is actively star-forming, whereas lower $\Delta(\rm Quench)$ means the galaxy has lower star formation rate. We color-code the galaxies in Figure \ref{fig:mass_size} with $\Delta(\mathrm{Quench})$. UDGs below the $1.5\sigma$ line have near zero or negative $\Delta(\mathrm{Quench})$, indicating that the UDG sample includes plenty of quenched galaxies compared with the UPG sample. Instead, the UPG sample is comprised of many star-forming galaxies at the high-mass end. 

% In Figure \ref{fig:mass_size}, we also show constant surface brightness lines in the left panel and constant mass surface density in the right panel. We will discuss how the definition of UDG could bias the quenched fraction later in Sec. \ref{sec:discussion}. 

\subsection{Abundance of mass-size outliers}\label{sec:n_udg}

\begin{figure*}
	\vbox{ 
		\centering
		\includegraphics[width=1\linewidth]{N_UDG_host_mass.pdf}
	}
    \caption{The abundance of UDGs as a function of host halo mass. \textit{Left}: We compile the UDG abundance measurements from the literature covering a wide range of host mass. The UDG abundance of our work $N_{\rm UDG} = 0.76\pm 0.04$ is shown as red square, the power-law relation from \citet{vdBurg2017} is shown in gray, and our fitting result is shown in pink. Our UDG abundance is consistent with ELVES but significantly higher than that in SAGA. After including more measurements at the lower halo mass end, the power-law is shallower than the relation reported in \citet{vdBurg2017}. \textit{Right}: We split the UDG sample into bins based on the host halo mass (triangles) and $g-i$ color (pentagons). The UDG abundance is higher for more massive host and redder host. }
    \label{fig:n_udg}
\end{figure*}

As demonstrated by many previous studies \citep[e.g.,][]{vdBurg2016,vdBurg2017,Roman2017a,Karunakaran2022b}, the average number of UDGs per host scales with host halo mass. While much literature focuses on finding UDGs in clusters and large groups, there are fewer constraints on the UDG abundance at the lower halo mass end. In this section, we calculate the UDG and UPG abundances of MW analogs, and compare with other surveys.

We define the UDG (UPG) abundance as the average number of UDGs (UPGs) per host galaxy. In our UDG sample, there are 412 UDGs associated with 258 hosts. After correcting for background contamination (see \S \ref{sec:bkg}) and completeness (see \S \ref{sec:comp_meas}), the UDG abundance of MW analogs is $N_{\rm UDG} = 0.76\pm 0.04$ per host. The UPG abundance in our sample is $N_{\rm UPG} = 0.61\pm 0.04$ per host. Here we neglect the fact that the number of satellites contained within the virial sphere is different from the number of satellite within a projected cylinder with virial radius (the so-called deprojection factor, \citealt{vdBurg2017}).


We compare our UDG abundance with other surveys focused on MW analogs. In the SAGA survey \citep{SAGA-II}, 6 satellite galaxies (out of 127 spectroscopy-confirmed satellites around 36 MW analogs) satisfy the definition of UDG. Therefore the UDG abundance in \citet{SAGA-II} is about $N_{\rm UDG}^{\rm SAGA}\approx 0.17\pm0.07$. In the ELVES survey \citep{CarlstenELVES2022}, there are 13 UDGs out of 358 satellites with secure distances ($P_{\rm sat} > 0.8$) in 30 MW analogs, leading to a UDG abundance of $N_{\rm UDG}^{\rm ELVES} \approx 0.43\pm0.12$. \citet{Roman2017b} identified 11 UDGs around three galaxy groups in the IAC Stripe 82 Legacy Survey \citep{Fliri2016}. Because both SAGA and ELVES select satellites based on distance measurements and do not correct for completeness, we interpret their UDG abundances as lower limits. We note that \citet{Roman2017b} do not estimate the background contamination fraction and correct for completeness. Because there is little literature on UPGs, we only compare our UPG abundance with ELVES. In ELVES, we take the 351 satellites with secure distances in 30 MW analogs and identify 36 UPGs associated with 20 hosts. Thus the UPG abundance fraction in ELVES is $N_{\rm UPG}^{\rm ELVES} = 1.20 \pm 0.20$, which is higher than our UPG abundance. 

We plot the UDG abundances of these surveys in the left panel of Figure \ref{fig:n_udg}, together with the results for larger galaxy groups and clusters \citep{Koda2015,Munoz2015,Roman2017a,Roman2017b,Janssens2017,vdBurg2017}. 
The power-law regression result from \citet{vdBurg2017} $N_{\rm UDG} \propto M_h^{1.11\pm 0.07}$ is shown in gray. The UDG abundance of this work is highlighted as the red square. As shown in Figure \ref{fig:n_udg}, the scatter in the $N_{\rm UDG}-M_h$ relation gets larger at the lower halo mass end, but this might be due to the small statistics in prior studies and the difference in completeness. Our UDG abundance is similar to the value in ELVES survey, but much higher than the value from SAGA. This might be explained by the different photometric depth of data used. As pointed out by \citet{CarlstenELVES2022,Font2022}, the SAGA survey reaches a surface brightness depth of $\sbeffr\approx 25\ \sbunit$, which is $\sim 2.5\ \sbunit$ shallower than the ELVES survey and this work. It is probable that SAGA missed a significant fraction of red and lower surface brightness galaxies. Our UDG abundance is marginally higher than the prediction from \citet{vdBurg2017} but is still consistent considering the large scatter of the power law. 

Taking all data points from literature (as shown in Figure \ref{fig:n_udg}), we use the orthogonal distance regression (ODR) to fit a power-law between $N_{\rm UDG}$ and host halo mass $M_h$ and find $N_{\rm UDG} \propto M_h^{0.96\pm 0.04}$, shown as the red dashed line in Figure \ref{fig:n_udg}. The slope of the power law is shallower than \citet{vdBurg2017} ($\beta=1.11\pm0.07$) but steeper than \citet{Roman2017b} ($\beta=0.85\pm0.05$). \citet{ManceraPina2018} survey eight clusters and find sublinear power laws. Recently \citet{Karunakaran2022b} perform a similar analysis including SAGA and ELVES and find slightly shallower power law of $\beta=0.87\pm0.07$. Since most individual studies do not have a statistical estimation for background contamination and completeness, the UDG abundances for large groups should be interpreted as lower limits. One major caveat in such studies is that most literature results are not completeness corrected and background subtracted, and these effects could significantly bias the results when combing them together. Therefore we caution that the power law index here should not be over-interpreted. 
% Among these studies, only \citet{vdBurg2017} applies a background subtraction and completeness correction to the raw counts.
% If completeness is applied, it is possible that the slope $\beta$ will be steeper than what we are showing in Figure \ref{fig:n_udg}. 

In the right panel of Figure \ref{fig:n_udg}, we split the UDG sample into different bins based on their host halo mass (shown as triangles) and host color (shown as pentagons). We find that the UDG abundance is slightly higher for hosts with higher halo mass and redder color, but the trend is stronger with host color. 
\vspace{1em}

Another interesting quantity is the fraction of satellites that are mass-size outliers in a group or cluster. This fraction represent how efficient an environment is at producing puffy satellites. We denote this quantity as the UDG (UPG) fraction hereafter, which characterizes the tail of the size distribution of satellites. In order to calculate this fraction, we assume that MW analogs have $\sim 6$ satellites with $M_\star > 10^{6.5}\ M_\odot$ \citep{CarlstenELVES2022}. This lower limit of stellar mass is roughly our detection limit (see Fig. \ref{fig:mass_size}). Then for our samples, the UDG (UPG) fraction is $f_{\rm UDG} \approx 0.15$ and $f_{\rm UPG} \approx 0.10$. In ELVES, among 351 satellites with secure distances in 30 MW analogs, we identify 13 UDGs associated with 10 hosts, and 36 UPGs associated with 20 hosts. Thus the UDG fraction in ELVES is $f_{\rm UDG}^{\rm ELVES} \approx 0.04$, the UPG fraction in ELVES is $f_{\rm UPG}^{\rm ELVES} \approx 0.10$. We also compare our results with results in the Virgo cluster. Taking the data from the Next Generation Virgo Cluster Survey (NGVS, \citealt{Ferrarese2020}), there are 15 UDGs and 24 UPGs out of 404 identified satellites with no completeness correction. Thus their UDG (UPG) fraction is $f_{\rm UDG}^{\rm Virgo} \approx 0.04$ and $f_{\rm UPG}^{\rm Virgo} \approx 0.06$. However, we note that \citet{Ferrarese2020} only survey the central 4 deg$^2$ of the Virgo cluster, which probably biases the UDG (UPG) fraction. It seems our UDG fraction is higher than both ELVES and NGVS. Our UPG fraction is consistent with ELVES but higher than NVGS. A systematic and careful comparison on UDG (UPG) fraction in different environments is needed in the future. 

\section{Discussion: robust identification of mass-size outliers}\label{sec:discussion}

% \subsection{}\label{sec:artifact}
% In this paper, we propose a robust definition of mass-size outliers (the UPGs). Here we contrast UPGs with UDGs to elaborate the fact that UPGs better represent mass-size outliers. 

% We also discuss the advantages and disadvantages of different definitions of low surface brightness galaxies including UDG and UPG. 

In Figure \ref{fig:mass_size}, we find that combining a hard physical size cut with a surface brightness cut carves out an interesting region of the mass-size plane that is complicated in two ways. On the one hand, there are many galaxies classified as UDGs that are not size outliers at $M_\star \sim 10^{7.5}\ M_\odot$ (highlighted as red stars in Fig. \ref{fig:mass_size}). On the other hand, the hard surface brightness cut corresponds to very different surface mass density between red and blue galaxies, making it hard to directly compare red and blue UDGs since they occupy different regions on the mass-size plane. For a given stellar mass and size (e.g., at $M_\star \sim 10^{7.5}\ M_\odot$), the hard surface brightness cut in the UDG definition preferentially removes blue galaxies because their surface brightnesses are higher and their $M_\star/L$ are lower. Thus, red UDGs are over-represented. If one calculates the quenched fraction of UDGs as a function of stellar mass, it will be biased high because of the constant surface brightness cut. 

It is worth emphasizing that the concept of UDG was originally proposed in a cluster context where UDGs are mostly red and quenched \citep[e.g.,][]{vanDokkum2015} and the color-$M_\star/L$ effect does not matter much. However, UDGs are not all red and blue UDGs are also found in less dense environments. Therefore, the concept of UDG might not be optimal for systematic studies of diffuse dwarf galaxies with a range of colors in different environments. A more physically-motivated definition for mass-size outliers is needed.

% These issues manifest themselves in the high quenched fractions of UDGs. 
% In Section \ref{sec:sample_construction}, we have already argued that the hard size cut in UDG definition does not map clearly onto the mass-size relation, and we propose identifying UPGs as outliers in the mass-size plane. Here we discuss another limitations in the UDG definition, namely the hard cut in surface brightness regardless of the galaxy color. We also discuss the advantages and disadvantages of different definitions of low surface brightness galaxies including UDG and UPG. 
% The above issues manifest themselves in the derived quenched fractions of UDGs. In Section \ref{sec:quench}, we find that the quenched fraction of UDGs is higher than the quenched fractions of UPGs and the satellite population of MW analogs. Furthermore, it almost remains a constant over $>1.5$ dex in stellar mass. 
% We argue that UDG's high and constant quenched fraction is merely an artifact of the definition. We start from Figure \ref{fig:mass_size}, where we show UDGs and UPGs on the mass-size plane color-coded by $\Delta(\mathrm{Quench}) = (g-i) - (g-i)_{Q}$ indicating how much a galaxy has been quenched (\S \ref{sec:quench}). 
% We note that the color-$M_\star/L$ relation are only invoked when converting the absolute magnitude to stellar mass. Thus when studying the quenched fraction on size-luminosity plane (e.g., g-band absolute magnitude), the artifact would disappear \citep[e.g., see][]{Danieli2019}. 

The concept of ultra-puffy galaxy is defined based on the average mass-size relation and thus alleviates some of the issues in the UDG definition. They are defined to be $1.5\sigma$ above the average mass-size relation in the dwarf galaxy regime. The specific value above the average mass-size relation can vary to probe different parts of the size distribution. We advocate studying the UPG population in both observations and simulations to explore samples with different ``puffiness'' (e.g., comparing UPG with $1.5\sigma$ with UPG with $2\sigma$). 


However, in order to construct a UPG sample, one needs to know the distance, the color, a color-$M_\star/L$ relation to convert observed magnitudes to stellar mass, and a mass-size relation and its scatter. None of these are easy to obtain and free from systematic errors. Both size and stellar mass could have large uncertainties in the LSB regime. One also needs to apply a surface brightness cut (such as $\sbeff > 23.0\ \sbunit$ in this work) to make sure the UPG sample is not contaminated by disk galaxies with high surface brightness due to the bulges. On the one hand, reliable mass-size relation and its scatter are not available for all mass ranges. \citet{ELVES-I} derive the mass-size relation below $M_\star \sim 10^{8.5}\ M_\odot$, whereas \citet{Lange2015} have relatively good constraints on the mass-size relation only above $M_\star \sim 10^{9}\ M_\odot$. The mass-size relation at $10^{8.5}\ M_\odot < M_\star < 10^{9}\ M_\odot$ is not well measured, nor is any dependence on color and morphology. In this work, we simply extrapolate the \citet{ELVES-I} mass-size relation to $10^9\ M_\odot$. At the same time, it is not clear whether the size distribution for a given mass can be well-described by a Gaussian distribution. Fig. 9 in \citet{ELVES-I} shows the residual in $\log(r_e)$ after fitting the mass-size relation, where the distribution of residuals is roughly Gaussian but skewed toward the large-size end. With enough statistics in the future, one might be able to characterize the shape of the large-size tail and define UPGs based on the percentiles of the size distribution \citep{Greene2022}. However, there are also other size definitions which help to reduce the scatter in the mass-size relation \citep[e.g.,][]{Miller2019,Mowla2019,Trujillo2020,Chamba2022} and can further benefit construct UPG samples. 

In the right panel of Figure \ref{fig:mass_size}, we also plot constant average surface mass density lines, which are steeper than the mass-size relation but have the same slope as constant surface brightness lines. Hence another way to define diffuse galaxies is to select galaxies with low surface mass density. This method also alleviates the artifact introduced by color-$M_\star/L$ relation on the mass-size plane. We defer exploration of this definition to future works.  


\section{Summary}\label{sec:summary}
In this work, we perform a search for low surface brightness galaxies in HSC PDR2 data and construct samples for mass-size outliers around MW analogs at $0.01 < z < 0.04$. Beside the ultra-diffuse galaxies, we propose studying ``ultra-puffy galaxies'' which is defined to be $1.5\sigma$ above the average mass-size relation. We conduct statistical analysis for mass-size outliers and argue that UPGs better represent the large-size tail of the dwarf galaxy population. We summarize our main findings and prospects below. 

\begin{enumerate}
    \item Using the HSC PDR2 data ($\sim 300\ \mathrm{deg}^{2}$), we conduct a systematic search for LSBGs using the method in \citet{Greco2018} complemented by our new deblending and modeling methods. Utilizing the non-parametric models generated by \code{scarlet}, we design a metric based on morphological parameters to effectively rule out false positives in the LSBG detection. We measure the structural properties of LSBGs using Spergel profiles and carefully characterize the measurement biases and uncertainties. The completeness of the search is also derived by injecting mock galaxies. Our LSBG search achieves high completeness compared with other searches and demonstrates the power of HSC data for LSB sciences. 
    
    \item By matching LSBGs with MW analogs in the NSA catalog, we construct samples of mass-size outliers including UDGs and UPGs. We display their distributions on the size-surface brightness plane (Figure \ref{fig:udg_upg_re_mu}) and the mass-size plane (Figure \ref{fig:mass_size}). The UPG sample contains more blue and higher surface brightness galaxies than the UDG sample. In contrast, the UDG sample includes many red galaxies below $1.5\sigma$ from the average mass-size relation.
    
    \item After correcting for background contamination, the UDG abundance in MW analogs is $N_{\rm UDG} = 0.76\pm 0.04$ per host and the UPG abundance is $N_{\rm UPG} = 0.61 \pm 0.04$ per host. Our UDG abundance is slightly higher than that in ELVES, but much higher than that in SAGA. Combining data from studies in denser environments, we find that the UDG abundance follows a sublinear power law with the host halo mass. The power law slope agrees with other studies. We caution that many other studies do not correct for completeness which could bias the results.
    
    \item For a MW analog, on average, about 15\% of its satellites are UDGs and about 10\% of its satellites are UPGs. The UDG and UPG fractions are slightly higher than those in ELVES and NGVS, which suffer from small number statistics due to their sample sizes. 
    
    \item We advocate studying UPG which is more physically motivated, does not introduce artifacts in the mass-size distribution, and better represents the large-size tail of dwarf galaxy population. 
% \end{itemize}
\end{enumerate}

This paper presents the UDG and UPG sample and explores their abundances. In our Paper II (to be submitted), we study their size and spatial distribution and their star formation status. This study is focused on a small subset of LSBG candidates matched with MW analogs. In future works, we would like to exploit the full sample to study interesting topics including redshift distributions, nucleation fractions, intrinsic shapes of LSBGs. We will also explore machine learning techniques to help us classify LSBGs and estimate structural parameters. With the upcoming LSST \citep{lsst2009,LSST2019}, we will be able to study the LSBG population with much greater statistics and gain new insights into the formation and evolution of mass-size outliers. 

\section*{Acknowledgment}
JL is grateful for discussions with Meng Gu, ChangHoon Hahn. The authors thank Yao-Yuan Mao for his tool for visualizing the image cutouts. 

J.H.K acknowledges the support from the National Research Foundation of Korea (NRF) grant, No. 2021M3F7A1084525, funded by the Korea government (MSIT).

The Hyper Suprime-Cam (HSC) collaboration includes the astronomical communities of Japan and Taiwan, and Princeton University. The HSC instrumentation and software were developed by National Astronomical Observatory of Japan (NAOJ), Kavli Institute for the Physics and Mathematics of the Universe (Kavli IPMU), University of Tokyo, High Energy Accelerator Research Organization (KEK), Academia Sinica Institute for Astronomy and Astrophysics in Taiwan (ASIAA) and Princeton University.  
Funding was contributed by the FIRST program from Japanese Cabinet Office, Ministry of Education, Culture, Sports, Science and Technology (MEXT), Japan Society for the Promotion of Science (JSPS), Japan Science and Technology Agency (JST), Toray Science Foundation, NAOJ, Kavli IPMU, KEK, ASIAA and Princeton University. Remy Joseph's work has been enabled by support from the research project grant ‘Understanding the Dynamic Universe’ funded by the Knut and Alice Wallenberg Foundation under Dnr KAW 2018.0067.

The authors are pleased to acknowledge that the work reported on in this paper was substantially performed using the Princeton Research Computing resources at Princeton University which is consortium of groups led by the Princeton Institute for Computational Science and Engineering (PICSciE) and Office of Information Technology's Research Computing.

\vspace{1em}
\software{\href{http://www.numpy.org}{\code{NumPy}} \citep{Numpy},
          \href{https://www.astropy.org/}{\code{Astropy}} \citep{astropy}, \href{https://www.scipy.org}{\code{SciPy}} \citep{scipy}, \href{https://matplotlib.org}{\code{Matplotlib}} \citep{matplotlib},
          \href{https://statmorph.readthedocs.io/en/latest/}{\code{statmorph}} \citep{statmorph},
        %   \href{https://halotools.readthedocs.io/en/latest}{\code{Halotools}} \citep{Hearin2017},
          \href{https://bdiemer.bitbucket.io/colossus/index.html}{\code{Colossus}} \citep{Colossus},
          \href{https://pmelchior.github.io/scarlet/}{\code{scarlet}} \citep{Melchior2018}, \href{https://github.com/dr-guangtou/unagi}{\code{unagi}}.
          }


\bibliography{citation}{}
\bibliographystyle{aasjournal}



\newpage
\appendix 


\section{LSBG detection}\label{ap:detection}
In this appendix, we provide a more detailed description on our LSBG search method and how it is different from \citetalias{Greco2018}.

\subsection{Bright source removal}
Bright sources and their associated LSB outskirts can mimic objects of interest and obstruct the detection of LSBGs. In this step, we replace pixels related to bright sources and small compact sources with sky noise. The bright sources and their diffuse outskirts are detected in the $i$-band by applying a high thresholding and a low thresholding to the image, respectively. We associate a diffuse light component with a bright source if more than 15\% of its pixels are above the high threshold. In this way, we generate a footprint of bright sources and their associated LSB components. Then for the $gri$-bands, we replace every pixel within this footprint with Gaussian noise, where the noise level is determined by masking out all detected sources in the image. 
    
In \citetalias{Greco2018}, the thresholds are set based on the noise level of the local sky. Thanks to the global sky subtraction in HSC PDR2, LSB features are well conserved after subtracting the sky and the sky noise level varies less. We therefore set the thresholds based on the characteristic surface brightness instead of a certain sigma value above the sky background. In this work, we set the high threshold to $\mu_{\rm high} = 22\ \sbunit$ to capture all bright sources above this surface brightness, and the low threshold to $\mu_{\rm low} = 24.5\ \sbunit$ to capture the associated diffuse light. Unlike in \citetalias{Greco2018}, we find it unnecessary to smooth the image prior to the thresholding based on our completeness tests (see \S \ref{sec:comp_meas},\S\ref{ap:comp_meas_unc}).   

After this step, there are still a number of small and compact LSB objects, which are typically marginally resolved galaxies, blended sources, or just pixels standing out because of noise fluctuation. We add an extra step to detect and remove them since our main goal is to detect extended LSBGs. We run \code{sep} on the ``cleaned'' images as described above. Based on the segmentation map, we generate a mask for sources smaller than $r_{\rm min} = 2\arcsec$, and pixels within this mask are also replaced by sky noise. This step significantly improves the purity of LSBG search. The values of ($\mu_{\rm high},\ \mu_{\rm low},\ r_{\rm min})$ are chosen by trial and error, but guided by the completeness tests. 
% This step effectively cleans out objects and features that hinder the detection of LSBGs. 
    
\subsection{Source Extraction}
We use \code{SourceExtractor} to detect sources on the ``cleaned'' images where the bright sources and small compact sources are replaced by sky noise. This step remains largely the same as in \citetalias{Greco2018}. The images are convolved with a Gaussian kernel of FWHM=$1\arcsec$ to enhance the contrast between LSBGs and sky background \citep[e.g.,][]{Irwin1985,Akhlaghi2015,Greco2018}. We take a mesh size of $43\arcsec$ (double the mesh size used in \citetalias{Greco2018} to allow bigger galaxies in our sample) to measure the local background and detect objects that are 0.7$\sigma$ per pixel above the local sky background. We also require the object to contain at least 100 contiguous pixels (equivalent to a square box with $1.7\arcsec$ on a side) to further remove small compact objects. We perform the detection in the $g$-band, but require that all sources are also detected in the $r$-band to exclude spurious detection and artifacts.
    
\subsection{Initial Sample Selection} 
We take the output catalog from \code{SourceExtractor} and remove those objects that are not likely to be LSBGs based on their sizes and colors. To be specific, we require objects to have $g$-band half-light radii (measured by \code{SourceExtractor}) greater than $r_{\rm min} = 2.0\arcsec$. We also require the measured colors to satisfy $-0.1 < g-i < 1.4$ and $|(g-r) - 0.7\cdot (g-i)| < 0.4$. This color box is relatively conservative with respect to the color distribution of the discovered LSBGs \citep[e.g.,][]{SAGA-I,Greco2018,Zaritsky2019,Tanoglidis2021}. 

Compared with \citetalias{Greco2018}, we add a new metric based on the morphology of sources to further remove compact point-like sources. For each source, we compute the ratio between the 2-D autocorrelation function (ACT) within an aperture of 5 pixels and the ACT within an annulus between 5 pixels and 9 pixels from the center. The ACT ratio essentially characterizes the peakiness of the source. Peaky sources have higher ACT ratios, thus we remove point-like sources by applying a cut on the ACT ratio $a_{\rm corr} < 2.5$. This step helps to improve the purity of our sample.

%%%%%%%%%%%%%%%%%%%%%%%%%%%%%%%%%%%%%%%%%%%%%
%%%%%%%%%%%%%%%%%%%%%%%%%%%%%%%%%%%%%%%%%%%%%
%%%%%%%%%%%%%%%%%%%%%%%%%%%%%%%%%%%%%%%%%%%%%

\section{Deblending details}\label{ap:deblending}
In \S\ref{sec:deblending}, we briefly introduce how we use \code{scarlet} to model the LSBG candidates in a non-parametric fashion and filter out false positives based on the structural and morphological parameters. In this appendix, we describe details of the implementation of vanilla \code{scarlet} which could help interested readers better understand the technique.

\subsection{Peak detection}\label{sec:peak}
We generate cutout images with a size of $1\arcmin$ in the $griz$-bands for each LSBG candidate in our initial sample. We then construct a detection image by taking an average of the four bands weighted by the inverse variance of each band. This detection image is considered to be deeper than the single-band images. 
Next, we run \code{sep} on the detection image using a threshold of 4$\sigma$ above the sky, a mesh size of 48 pixels ($8\arcsec$), and a kernel size of 3 pixels. This step identifies extended sources in the detection image. However, there are still faint and compact peaks not detected. We apply a wavelet decomposition to the detection image \citep{Starck2015} and only keep the high spatial frequency components (also see \citealt{Zaritsky2019} for an example of using wavelet filtering in the context of LSBGs). Another round of \code{sep} is run on this high-frequency image using a detection threshold of 2.5$\sigma$, a mesh size of 24 pixels, and a kernel size of 3 pixels. This step detects many compact sources that were not included in the previous step. In the end, we combine the two detection catalogs and remove duplicates. 

\subsection{Model initialization and optimization}
After the peak detection step, we need to decide which peaks to model. It is not necessary to model all detected peaks because the deblending step is designed to model the sources only in the vicinity of the target LSBG candidate. Therefore, the size of the target object determines which peaks are relevant. Nevertheless, it is also important to choose the appropriate model for each source and initialize them as best as possible. In practice, we draw a square bounding box around the target object and model all the peaks inside this box. The bounding box is in turn determined as we initialize the model for the target object. We describe the procedures as follows. 

First, in vanilla \code{scarlet}, objects can be modeled with different types of models\footnote{\url{https://pmelchior.github.io/scarlet/1-concepts.html\#Source}} including point source, single-extended source, multi-extended source, compact source, and flat-sky source. The morphology image of a point source is simply the normalized PSF model. The single-extended source has a morphology image that follows positivity and monotonicity constraints. The multi-extended source is a combination of two or more co-centered single-extended sources, making it possible to model galaxies with more complex structures and capture any color gradients. The compact source is a single-extended source initialized using the morphology image of a point-source, which encourages the model to be compact, but still allows the morphology image to be extended. The flat-sky source has uniform color and morphology within the bounding box.

For modeling the LSBG candidates, we use an extended source model with two components such that the model is able to capture the galaxy structure and color gradient. The target source is initialized in the following way. We convolve the detection image with a circular Gaussian kernel with $\sigma=1.5$ pixels to boost the contrast between the signal and the sky noise. After smoothing, the LSB outskirts of the target galaxy become more prominent, which helps in determining the bounding box and initializing the morphology image. We take the smoothed image and threshold it with $0.1\sigma$ to remove the sky noise. The initial morphology image $S_i$ is determined by constructing a symmetric and monotonic approximation to the smoothed image around the target object. The initial SED vector $A_i$ is set to be the sum of the morphology matrices $S_i$ in each band. As a byproduct of initialization, we get a bounding box from $S_i$ which indicates the extent of the target object. Consequently, the size of the bounding box depends on the smoothing kernel and the threshold. We choose the above values by trial and error such that the box is not so large as to include many irrelevant peaks, but not so small as to lose a significant fraction of LSB outskirts of the target galaxy. 

Then we take all peaks within the bounding box of the target galaxy and initialize them in the same way as described above. Recall that we have two rounds of peak detection (Sec \ref{sec:peak}), focusing on extended and compact sources respectively. The extended objects detected in the first peak detection step are modeled as single-extended sources. For compact objects that are only detected in the second peak detection, we model them as point sources if their FWHM $<$ 5 pixels, otherwise as compact-extended sources. We also add a flat-sky source to model the local sky around the target. This is helpful for situations where an object overlaps with the LSB outskirts of a bright galaxy (e.g., the bottom panel in Figure \ref{fig:vanilla_scarlet_demo}). We note that whether adding a flat-sky source could significantly change the size and total magnitude of the \code{scarlet} model.

Although we only model peaks within the bounding box of the target, the scattered light from nearby bright stars and galaxies could bias the modeling of the sources within the bounding box (e.g., the yellow star in the top left panel of Figure \ref{fig:vanilla_scarlet_demo}). We match our field with the Gaia catalog \citep{GAIA2016,GAIA2018} and mask out stars outside of the bounding box. Since we have already detected peaks throughout the whole cutout image, we also generate a mask for objects outside the bounding box to reduce the impact of scattered light from bright galaxies on the modeling of the flat-sky source. 

The optimization process uses the adaptive proximal gradient method \citep{Melchior2019}, which is a robust method for optimization with constraints. The model is considered to be converged when the relative changes of parameters are smaller than \code{e\_rel\,=\,2e-4}. Typically, convergence is achieved after $\sim 50$ steps of optimization and the whole modeling takes about 40s for a typical LSBG. It takes longer when the target galaxy has large angular size since more peaks are modeled. We note that \code{scarlet} only find the maximum likelihood estimation of the model instead of deriving the full posteriors, thus a good initialization becomes especially important for fast convergence. 

%%%%%%%%%%%%%%%%%%%%%%%%%%%%%%%%%%%%%%%%%%%%%
%%%%%%%%%%%%%%%%%%%%%%%%%%%%%%%%%%%%%%%%%%%%%
%%%%%%%%%%%%%%%%%%%%%%%%%%%%%%%%%%%%%%%%%%%%%

\section{Spergel profiles}\label{ap:spergel}
\begin{figure*}[htbp!]
	\vbox{ 
		\centering
		\includegraphics[width=0.75\linewidth]{spergel_sersic_calib.pdf}
	}
    \caption{The correspondence between the \sersic{} profile \eqref{eq:sersic} and the Spergel profile \eqref{eq:spergel}. We fit a Spergel profile to \sersic{} while fixing the total luminosity and half-light radius. The left panel shows the best-fit Spergel index $\nu$ as a function of \sersic{} index $n$. In the right panel, two \sersic{} profiles (solid) and their best-fit Spergel profiles (dash-dotted) are shown. Spergel profile approximates \sersic{} well for small \sersic{} indices.  
    }
    \label{fig:spgl_calib}
\end{figure*}

In this appendix, we demonstrate that the Spergel profile can approximate \sersic{} profile and profile a lookup table for the correspondence between \sersic{} index $n$ and Spergel index $\nu$.

The surface brightness of a Spergel profile has the form \citep{Spergel2010}:
\begin{equation}
    \label{eq:spergel}
    I_\nu(r) = \frac{c_{\nu}^{2} L_{0}}{2\pi r_{0}^{2}} f_{\nu}\left(\frac{c_{\nu} r}{r_{0}}\right),
\end{equation}
where 
\begin{equation}
    f_{\nu}(u)=\left(\frac{u}{2}\right)^{\nu} \frac{K_{\nu}(u)}{\Gamma(\nu+1)},
\end{equation}
and $K_\nu(u)$ is the Modified Bessel function of the second kind. The half-light radius is $r_0$, the total luminosity is $L_0$, and $c_\nu$ satisfies the equation $(1 + \nu)f_{\nu + 1}(c_\nu) = 1/4$. The Spergel profile has a simple analytical expression in Fourier space, making it easy to convolve with a PSF.

The surface brightness of a \sersic{} profile follows \citep{Sersic1963,Graham2005}:
\begin{equation}\label{eq:sersic}
    I(r)=I_{\mathrm{e}} \exp \left\{-b_{n}\left[\left(\frac{r}{r_{\mathrm{e}}}\right)^{1 / n}-1\right]\right\},
\end{equation}
where $r_e$ is the half-light radius, $I_e$ is the surface brightness at $r=r_e$, $n$ is the \sersic{} index. The value of $b_n$ satisfies $\Gamma(2 n)=2 \gamma\left(2 n, b_{n}\right)$, where $\gamma(a, x)$ is the incomplete gamma function. According to \citet{Graham2005}, the total luminosity of a \sersic{} profile is given by 
\begin{equation}\label{eq:sersic_lum}
    L_0 = I_{e} r_{e}^{2}\, 2 \pi n\, e^{b_{n}} \left(b_{n}\right)^{-2 n} \Gamma(2 n).
\end{equation}

To study the correspondence between \sersic{} and Spergel profiles, we generate \sersic{} profiles with different \sersic{} indices, and try to fit Spergel profiles to \sersic{} ones. The \sersic{} index ranges from $n=0.5$ to $n=4.5$, and both profiles are normalized using $r_e$. For each \sersic{} profile, we calculate the total luminosity $L_0$ according to \eqref{eq:sersic_lum} and plug it into the Spergel profile \eqref{eq:spergel} as a fixed value. Therefore, only the Spergel index $\nu$ is allowed to vary during the fitting. The best-fit Spergel index $\nu$ as a function of \sersic{} index $n$ is shown in the left panel of Figure \ref{fig:spgl_calib}. As two examples, we show two \sersic{} profiles (solid) and their best-fit Spergel profiles (dash-dotted) in the right panel. A Spergel profile with $\nu=0.5$ is exactly an exponential profile with $n=1$. The de Vaucouleurs profile \citep{deVaucouleurs1948} with $n=4$ can be approximated by a Spergel profile with $\nu=-0.74$. 

As shown in Figure \ref{fig:spgl_calib}, a \sersic{} profile with small \sersic{} index ($0.5 < n < 1.5$) can be well-approximated by a Spergel profile, although the Spergel profile seems to be more extended than \sersic{} in the outskirts. For a \sersic{} profile with high \sersic{} index, the approximation gets worse at both small and large radii. Overall, the Spergel profile is a good approximation to \sersic{} from $n\approx 0.5$ to $n\approx 4.5$. It is well-known that the light profiles of low-mass galaxies are quite flat and can be described using \sersic{} profiles with $0.5 < n < 1.5$ \citep[e.g.,][]{vanDokkum2015,Lange2015,Greco2018,Zaritsky2021,ELVES-I}. It is thus reasonalbe to use Spergel profiles to model the LSBGs (\S\ref{sec:modeling}) and enjoy its convenience in Fourier space. 

\section{Completeness and Measurement Uncertainty}\label{ap:comp_meas_unc}
\subsection{Completeness}
The total completeness is a multiplication between the detection completeness and deblending completeness. Below we describe how we generate mock galaxies and derive the recovered fraction. 

For the detection completeness, we inject $\sim 700,000$ mock galaxies with single \sersic{} light profiles into the coadd images\footnote{We have also done extensive tests on injecting mock galaxies into the raw images and going through the entire data reduction pipeline. This is very expensive in terms of both CPU time and disk space, since we must run the full \code{hscPipe}. However, we find no noticeable difference in the completeness between this method and the direct injection to coadd images.}. To resemble the real LSBG population in \citetalias{Greco2018}, we generate mock galaxies following uniform distributions in size ($2\arcsec \leqslant r_{e} \leqslant 21\arcsec$), surface brightness ($23 \leqslant \overline{\mu}_{\rm eff}(g) \leqslant 28.5\ \mathrm{mag\ arcsec^{-2}}$), \sersic{} index ($0.8 < n < 1.2$), and ellipticity ($0 < \varepsilon < 0.6$). They are randomly assigned to have a blue ($g-i=0.47,\ g-r=0.32$), medium ($g-i=0.64,\ g-r=0.43$), and red ($g-i=0.82,\ g-r=0.56$) color with equal chance. Then we run the detection step as described in \S \ref{sec:detection} and cross-match the detection catalog with the input mock galaxy catalog to calculate completeness. We split the size and surface brightness range into 15 bins with $\Delta r_e = 0.86\arcsec$, $\Delta \sbeff = 0.33\ \sbunit$, and interpolate over bins using an isotropic Gaussian kernel with $\sigma = 0.5$. We notice that the size and surface brightness shown in Figure \ref{fig:completeness} are all the intrinsic values for the mock galaxies, which are different from the measured ones. We find negligible dependence of detection completeness on \sersic{} index, color, and ellipticity. Therefore, we neglect the dependence of detection completeness on parameters other than size and surface brightness hereafter.\footnote{We performed a smaller set of image simulation where we extend the ellipticity range and also simulated galaxies with additional structures (such as star-forming clumps). We find the completeness declines for $\varepsilon > 0.6$, suggesting that edge-on disk galaxies may be missing from our sample. Please see \citet{Greene2022} for more details.}

For the deblending completeness, we inject 5,000 mock \sersic{} galaxies into the coadd and run vanilla \code{scarlet} on them. The mock galaxies follow the same uniform distribution in size, surface brightness, ellipticity, \sersic{} index as for deriving the detection completeness, but follow a Gaussian distribution in color: $g-i \sim \mathcal{N}(0.6, 0.2^2),\ g-r = 0.7 \cdot (g-i) + \mathcal{N}(0, 0.03^2)$. We measure the size and surface brightness on the scarlet models of the mock galaxies, then we apply the deblending cuts. We also find that the deblending cut mainly depends on the size and surface brightness. 

\subsection{Measurement bias and uncertainty}\label{sec:meas_unc}

Due to the low surface brightness nature of LSBGs, their size, magnitude, surface brightness and shape are hard to characterize and are typically associated with large uncertainties. \citet{Haussler2007} use mock single-\sersic{} galaxies and parametric fitting codes to demonstrate that the estimated size and total magnitude are sensitive to local sky estimation and masking of neighboring objects. In the low surface brightness regime, the measured size and total magnitude can be quite biased and have large uncertainties. Therefore, measurement bias and uncertainty must be considered when studying the properties of LSBGs. \citet{Zaritsky2021,Zaritsky2022} characterize their measurement bias and uncertainty by injecting mock \sersic{} galaxies into the coadd and compare the recovered properties with the truth. \citet{Tanoglidis2022ICML} recently propose a new method to estimate the measurement error using a Bayesian Neural Network. In this paper, we simply take the former method because of its simplicity. 

In order to test how well we recover the photometric and structural parameters in the our measurement (Sec \ref{sec:modeling}), we take the 5,000 mock \sersic{} galaxies used for computing the deblending completeness and model them using the Spergel light profile. We model the bias and uncertainty in size, surface brightness, total magnitude, and color as a function of other parameters including size, surface brightness, and shape. However, we do not find any significant dependence of the bias on color, ellipticity, and Spergel index. Thus we just model the bias and uncertainty as a function of the \textit{measured} angular size $r_e$ and surface brightness $\sbeff$.

Given the size and surface brightness of our simulated galaxies, we set the range of the observed size and surface brightness to be $r_e\in[1\arcsec, 15\arcsec],\ \sbeff\in[23, 29]\,\sbunit$. We then split the observed $r_e-\overline{\mu}_{\rm eff}(g)$ plane using an $8\times 8$ grid, and calculate the mean bias $\Delta X = X_{\rm truth} - X_{\rm meas}$ within each bin, where $X=\{\sbeff,\ g-i,\ g-r\}$. For $r_e$, we calculate the relative bias $\Delta r_e / r_e$ instead because it is less dependent on the angular size. Then we interpolate over the grid using a multi-quadratic kernel\footnote{\url{https://docs.scipy.org/doc/scipy/reference/generated/scipy.interpolate.RBFInterpolator.html}}. Unlike \citet{Zaritsky2021} where they fit models in 4-D space, we find interpolation works well enough in 2D and do not use polynomial fitting to avoid meaningless results outside of the fitting range. We emphasize that the bias and uncertainty are modeled to be functions of the measured properties, not intrinsic ones, such that we can correct for bias based on our measurements. 

For LSBGs in our sample, we apply corrections for the bias using the interpolated bias terms. We first correct for the bias in size, $g$-band average surface brightness, $g-r$ and $g-i$ colors. Then we calculate the $g$-band total magnitude following $m_g = \overline{\mu}_{\rm eff}(g) - 2.5\log(2\pi r_e^2)$. The magnitudes and surface brightnesses in other bands are derived using $g$-band magnitude, surface brightness, and colors. In this way, we apply a self-consistent correction for the measurement biases to the data. 
The measurement uncertainty consists of a statistical uncertainty which is determined by the shape of the likelihood (posterior) surface, and a systematic uncertainty which is related to various factors including sky subtraction, neighboring contamination, etc. Unlike other parametric modeling codes such as \code{imfit} \citep{imfit} and \code{the Tractor} \citep{Lang2016}, \code{scarlet} does not explore the full posterior space but rather finds one optimal solution. Thus we have no access to the statistical error on the measured properties from \code{scarlet}. Fortunately, by comparing the recovered properties of mock galaxies with the truth, we can empirically estimate the measurement uncertainty without knowing the impact of each factor. Following the same method as that of calculating the bias, in each bin we compute the $1\sigma$ standard deviation of the difference between the truth and the bias-corrected measurement, then we interpolate over the grid. The measurement errors $\sigma(X)$ are shown in Figure \ref{fig:meas_err} as contours, and they have the same units as the biases. We set minimum uncertainties to be $\sigma(r_e) \geqslant 0.3,\ \sigma(\overline{\mu}_{\rm eff}) \geqslant 0.05,\ \sigma(g-i) \geqslant 0.05$ to avoid meaningless uncertainty due to small statistics.

% \section{Measurement Error}\label{ap:meas_error}

% We characterize the quality of the Spergel modeling by injecting mock \sersic{} galaxies into the cutout and compare the recovered properties with the truth, as shown in Appendix \ref{ap:meas_error}. Overall speaking, the measurement agrees with the truth quite well. However, the size and the total flux of galaxies below $\overline{\mu}_{\rm eff} (g) > 27$ is under-estiamted. We also characterize the bias and the scatter in measurement as a function of size and surface brightness. We apply the bias correction to the measurement of LSBGs, and incorporate the measurement error into the science figures. 

% vdB 16 doesn't consider whether GALFIT gives the corrrect $R_e$, when deriving the completeness (recovered fraction)

% Remember to ref SMUDGES papers here.


\section{UDG and UPG Catalogs}
\onecolumngrid 

\begin{table}
\caption{UDG (UPG) catalog description} 
\label{tab:catalog}
\begin{center}
\begin{tabular}{l l l}
\hline\hline
% \multicolumn{3}{c}{Table: LSBGs (781 rows)}                 \\
% \hline
Column Name      & Unit    & Description                    \\
\hline
ID                       &         & Unique LSBG ID \\
ra                       & deg     & Right ascension (J2000) \\
dec                      & deg     & Declination (J2000) \\
$r_e$         & arcsec  & Circularized effective radius  \\
$\sigma(r_e)$ & arcsec  & Uncertainty of $r_e$ \\
$\overline{\mu}_{\mathrm{eff}}(g)$               & $\sbunit$ & $g$-band average surface brightness within $r_e$ \\
$\sigma(\overline{\mu}_{\mathrm{eff}}(g))$       & $\sbunit$ & Uncertainty of $\overline{\mu}_{\mathrm{eff}}(g)$           \\
% $\overline{\mu}_{\mathrm{eff}}(r)$               & $\sbunit$ & $r$-band average surface brightness within $r_e$ \\
% $\sigma(\overline{\mu}_{\mathrm{eff}}(r))$       & $\sbunit$ & Uncertainty of $\overline{\mu}_{\mathrm{eff}}(r)$           \\
% $\overline{\mu}_{\mathrm{eff}}(i)$               & $\sbunit$ & $i$-band average surface brightness within $r_e$ \\
% $\sigma(\overline{\mu}_{\mathrm{eff}}(i))$       & $\sbunit$ & Uncertainty of $\overline{\mu}_{\mathrm{eff}}(i)$           \\
$m_g$                    & mag     & $g$-band apparent magnitude     \\
$\sigma(m_g)$            & mag     & Uncertainty of $m_g$            \\
% $m_r$                    & mag     & $r$-band apparent magnitude     \\
% $\sigma(m_r)$            & mag     & Uncertainty of $m_r$            \\
% $m_i$                    & mag     & $i$-band apparent magnitude     \\
% $\sigma(m_i)$            & mag     & Uncertainty of $m_i$            \\
$g-r$                    & mag     & $g-r$ color                     \\
$\sigma(g-r)$            & mag     & Uncertainty of $g-r$ color      \\
$g-i$                    & mag     & $g-i$ color                     \\
$\sigma(g-i)$            & mag     & Uncertainty of $g-i$ color      \\
$\nu$                    &         & Spergel index              \\
$\varepsilon$            &         & Ellipticity                     \\
$A_g$                    & mag     & $g$-band Galactic extinction \\
$A_r$                    & mag     & $r$-band Galactic extinction \\
$A_i$                    & mag     & $i$-band Galactic extinction \\
$\log\ M_\star$ & $M_\odot$ & Stellar mass of UDG (UPG) \\
comp & & Completeness \\
weight & & Color-dependent weight (see Section \ref{sec:bkg}) \\
host\_name & & Name of the host galaxy \\
host\_ra & & Right ascension (J2000) of the host galaxy \\
host\_dec & & Declination (J2000) of the host galaxy \\
host\_log\_m\_star & $M_\odot$ & Host stellar mass\\
host\_r\_vir & kpc & Virial radius of the host galaxy \\
host\_g\_i & mag & $g-i$ color of the host galaxy \\
host\_z &  & Redshift of the Host galaxy \\
sep\_to\_host & deg & Angular separation between the host and UDG (UPG)\\
\hline\hline
\end{tabular}
\end{center}
\tablecomments{
These tables are published in their entirety in machine-readable format.
Magnitudes are on the AB system and have been corrected for measurement biases and Galactic
extinction. The information of the host galaxies is from NASA-Sloan Atlas. We provide Galactic extinction corrections, which are derived from
the \citet{Schlafly2011} recalibration of the \citet{SFD1998} dust maps. 
}
\end{table}

\end{CJK*}
\end{document}